\documentclass[a4paper]{article} 
\input{head}
\newcommand{\pow}[2]{#1^{#2}}
\newcommand{\supra}[1]{\textsuperscript{#1}}
\begin{document}

%-------------------------------
%	TITLE SECTION
%-------------------------------

\fancyhead[C]{}
\hrule \medskip % Upper rule
\begin{minipage}{0.295\textwidth} 
\raggedright
\footnotesize
Fausto David Hernández Jasso \hfill\\   
@FaustoJH \hfill\\
fausto.david.hernandez.jasso@ciencias.unam.mx
\end{minipage}
\begin{minipage}{0.4\textwidth} 
\centering 
\large 
Lógica Computacional I\\ 
\normalsize 
Introducción y Sintaxis de la Lógica de Proposiciones\\ 
\end{minipage}
\begin{minipage}{0.295\textwidth} 
\raggedleft
\today\hfill\\
\end{minipage}
\medskip\hrule 
\bigskip
\section{¿Por qué estudiar lógica computacional?}
\noindent
Porque la lógica en las ciencias de la computación, en cierto sentido es el \textbf{cálculo de la computación}
el cuál es el fundamento matemático para tratar la información y razonar acerca del comportamiento de programas.
\section{¿Por qué necesitamos la formalización del razonamiento correcto?}
\noindent
La lógica ha sido pieza clave para estructurar el pensamiento y el razonamiento:
\begin{itemize}
    \item Dar un fundamento a las matemáticas.
    \item Eliminar errores del razonamiento.
    \item Encontrar una forma eficiente para llegar a una justificación de una conclusión, dada cierta información en 
    forma de premisas.
\end{itemize}
\section{Argumentos lógicos}
\subsection{Definición}
\noindent
Es una colección finita de afirmaciones \textit{(proposiciones)} dividida en premisas y conclusión.
\subsection{Acerca de las premisas y de la conclusión}
\noindent
Las premisas y la conclusión debe ser susceptibles de recibir un valor de verdad. El argumento lógico puede ser
correcto o incorrecto.
\subsection{Validez de un argumento}
\noindent
Un argumento es correcto o válido si suponiendo que sus premisas son ciertas, entonces necesariamente la 
conclusión también lo es.
\subsection{Ejemplo}
\subsubsection{La isla de los caballeros y los bribones}
\noindent
En la isla de los caballeros y bribones sólo hay dos clases de habitantes, los caballeros que siempre dicen 
la verdad y los bribones que siempre mienten.
\newline
Un náufrago llega a la isla y encuentra dos habitantes: A y B. El habitante A afirma: Yo soy un bribón o 
B es un caballero. El acertijo consiste en averiguar cómo son A y B.
\subsubsection{Lógica}
\noindent
\begin{align*}
    p &:= \text{A es bribón} \\
    q &:= \text{B es un caballero} \\
    s & := p \lor q 
\end{align*}
\(s\) es lo que dijo A.
\newline
Supongamos que \(\mathcal{I}\left(p\right) = \top\)
\newline 
Así \(\mathcal{I}\left(s\right) = \bot\)
\newline 
Entonces \(\mathcal{I}\left(\neg s\right) = \top\)
\newline
Como \(\neg s \equiv \neg p \land \neg q\)
\newline 
Consecuentemente \(\mathcal{I}\left(\neg p \land \neg q\right) = \top\)
\newline 
Lo que implica que se cumpla
\begin{align*}
    \mathcal{I}\left(\neg p\right) &= \top \\
    \mathcal{I}\left(\neg q\right) &= \top \\
\end{align*}
Por lo tanto concluimos que A:
\[
    \mathcal{I}\left(p\right) = \bot
\]
Por lo tanto hemos llegado a una contradicción, ya que A no puede ser bribón y no serlo al mismo tiempo.
\newline 
Supongamos que \(\mathcal{I}\left(p\right) = \bot\)
\newline 
Así \(\mathcal{I}\left(s\right) = \top\)
\newline
Como \( s \equiv p \lor q\)
\newline 
Así tenemos que \(\mathcal{I}\left(p\right) = \bot\) y \(\mathcal{I}\left(q\right) = \top\)
\newline 
En éste caso A no es bribón y B es un caballero, entonces concluimos que:
\begin{itemize}
    \item A es un caballero.
    \item B es un caballero.
\end{itemize} 
\subsection{Características de un argumento lógico}
\begin{itemize}
    \item Involucran individuos.
    \item Los individuos que involucran tienen propiedades.
    \item Una proposición es una oración que puede calificarse como verdadera o falsa y habla de las propiedades de los individuos.
    \item Se forman medinate proposiciones, clasificadas como premisas y conclusión del argumento.
    \item Las porposiciones pueden ser compuestas.
    \item Puede ser correcto o incorrecto.
\end{itemize}
\section{Sistema lógico}
\noindent
Cualquier sistema lógico consta al menos de los siguientes tres componentes:
\begin{itemize}
    \item Sintaxis.
    \item Semántica.
    \item Teoría de la prueba.
\end{itemize}
\subsection{Sintaxis}
\noindent
Lenguaje formal que se utilizará como medio de expresión.
\subsection{Semántica}
\noindent
Mecanismo que proporciona significado al lenguaje formal dado por la sintaxis.
\subsection{Teoría de la prueba}
\noindent
Se encarga de decidir la correctud de un argumento lógico por medios puramente sintácticos.
\subsection{Propiedades}
\subsubsection{Consistencia}
\noindent
No hay contradicciones
\subsubsection{Correctud}
\noindent
Las reglas del sistema no pueden obtener una inferencia falsa a partir de una verdadera.
\subsubsection{Completud}
\noindent
Todo lo verdadero es demostrable.
\section{Lógica proposicional}
\subsection{Proposición}
\noindent
Es un enunciado que puede calificarse como verdadero o falso.
\subsection{Lenguaje \texttt{PROP}}
\begin{itemize}
    \item Símbolos o variables proposicionales \textit{(número infinito)}: \(p_{1}, \dotsc, p_{n}, \dotsc\)
    \item Constantes lógicas: \(\top, \bot\)
    \item Operadores lógicos: \(\neg, \land, \lor, \rightarrow, \leftrightarrow \)
    \item Símbolos auxiliares: \(\left(\right)\)
\end{itemize}
El conjunto de expresiones o fórmulas atómicas, denotado como \texttt{ATOM} consta de:
\begin{itemize}
    \item Variables proposicionales: \(p_{1}, p_{2}, \dotsc, p_{n}, \dotsc\)
    \item Constantes lógicas: \(\top, \bot\)
\end{itemize}
\subsection{Expresiones en el lenguaje \texttt{PROP}}
\subsubsection{Definición recursiva}
\begin{itemize}
    \item Si \(\varphi \in \text{\texttt{ATOM}}\) entonces \(\varphi \in \text{\texttt{PROP}}\).
    \item Si \(\varphi \in \text{\texttt{PROP}}\) entonces \(\left(\neg \varphi\right) \in \text{\texttt{PROP}}\).
    \item Sí \(\varphi, \psi \in \text{\texttt{PROP}}\) entonces \(\left(\varphi \land \psi \right), \left(\varphi \lor \psi \right), \left(\varphi \rightarrow \psi \right), \left(\varphi \leftrightarrow \psi \right) \in \text{\texttt{PROP}}\).
    \item Son todas.
\end{itemize}
\subsubsection{Backus-Naur}
\begin{align*}
    VarP &::= p_{1} \ | \ p_{2} \ | \ \dotsc \ | \ p_{n} \ | \ \dotsc  \\
    \varphi, \psi &::= VarP \ | \ \top \ | \ \bot \ | \ \left(\neg \varphi\right)  \ | \ \left(\varphi \land \psi \right) \ | \ \left(\varphi \lor \psi \right)  \ | \ \left(\varphi \rightarrow \psi \right)  \ | \ \left(\varphi \leftrightarrow \psi \right)  \ |
\end{align*}
\subsection{Precedencia y asociatividad de los operadores lógicos}
Precedencia de mayor a menor
\begin{itemize}
    \item \(\neg\)
    \item \(\lor, \land\)
    \item \(\rightarrow\)
    \item \(\leftrightarrow\)
\end{itemize}
Operadores asociativos:
\begin{itemize}
    \item \(\neg\)
    \item \(\lor, \land\)
    \item \(\leftrightarrow\)
\end{itemize}
El operador \(\rightarrow\) asocia hacia la derecha.
\newline 
Ejemplificando lo anterior la expresión
\[
    \varphi_{1} \rightarrow \varphi_{2} \rightarrow \varphi_{3}  
\]
Queda asociada como:
\[
    \varphi_{1} \rightarrow \left(\varphi_{2} \rightarrow \varphi_{3}\right)  
\]
\subsection{Eliminación de paréntesis innecesarios}
\noindent
Notemos que:
\[
     p \land q \rightarrow \neg q \lor s  
\]
es igual a
\[
    \left(p \land q\right) \rightarrow \left(\left(\neg q \right) \lor s\right)  
\]
Pero ¿Cómo podemos obtener la primera sí nos dan la última?
\newline 
Simplemente debemos de ir quitando paréntesis guiándonos por la precedencia de operadores
\begin{align*}
    \left(p \land q\right) &\rightarrow \left(\left(\neg q \right) \lor s\right) &\\
    \left(p \land q\right) &\rightarrow \left(\neg q \lor s\right) &\text{Eliminamos el paréntesis del operador \(\neg\)}\\
    p \land q &\rightarrow \neg q \lor s &\text{Eliminamos el paréntesis del operador \(\lor\) y \(\land\)}
\end{align*}
Veamos \(\left(p \rightarrow q \land r\right) \leftrightarrow \left(s \lor \left(\neg t\right)\right)\)
\begin{align*}
    \left(p \rightarrow q \land r\right) &\leftrightarrow \left(s \lor \left(\neg t\right)\right) &\\
    \left(p \rightarrow q \land r\right) &\leftrightarrow \left(s \lor \neg t\right) &\text{Eliminamos el paréntesis del operador \(\neg\)}\\
    \left(p \rightarrow q \land r\right) &\leftrightarrow s \lor \neg t &\text{Eliminamos el paréntesis del operador \(\lor\)}\\
    p \rightarrow q \land r &\leftrightarrow s \lor \neg t &\text{Eliminamos el paréntesis del operador \(\rightarrow\)}\\
\end{align*}
Veamos \(\left(\left(p \lor q\right) \rightarrow r\right) \leftrightarrow \left(\left(\neg r\right) \rightarrow \left(\neg\left(p \lor q\right)\right) \right)\)
\begin{align*}
    \left(\left(p \lor q\right) \rightarrow r\right) &\leftrightarrow \left(\left(\neg r\right) \rightarrow \left(\neg\left(p \lor q\right)\right) \right) & \\
    \left(\left(p \lor q\right) \rightarrow r\right) &\leftrightarrow \left(\neg r \rightarrow \left(\neg\left(p \lor q\right)\right) \right) & \text{Eliminamos el paréntesis del operador \(\neg\)} \\
    \left(\left(p \lor q\right) \rightarrow r\right) &\leftrightarrow \left(\neg r \rightarrow \neg\left(p \lor q\right) \right) & \text{Eliminamos el paréntesis del operador \(\neg\)} \\
    p \lor q \rightarrow r &\leftrightarrow \left(\neg r \rightarrow \neg\left(p \lor q\right) \right) & \text{Eliminamos el paréntesis del operador \(\rightarrow\)} \\
    p \lor q \rightarrow r &\leftrightarrow \neg r \rightarrow \neg\left(p \lor q\right) & \text{Eliminamos el paréntesis del operador \(\rightarrow\)}
\end{align*}
Veamos \(\neg\left(\left(\left(p \land \left(p \lor \left(\neg q\right)\right)\right) \land q \right) \land p\right)\)
\begin{align*}
    &\neg\left(\left(\left(p \land \left(p \lor \left(\neg q\right)\right)\right) \land q \right) \land p\right) & \\
    &\neg\left(\left(\left(p \land \left(p \lor \neg q\right)\right) \land q \right) \land p\right) & \text{Eliminamos el paréntesis del operador \(\neg\)}\\
    &\neg\left(\left(p \land \left(p \lor \neg q\right) \land q \right) \land p\right) & \text{Eliminamos el paréntesis del operador \(\land\)}\\
    &\neg\left(p \land \left(p \lor \neg q\right) \land q \land p\right) & \text{Eliminamos el paréntesis del operador \(\land\)}\\
\end{align*}
Veamos \(\left(\neg s\right) \rightarrow \left(\left(\neg t\right) \land \neg \left(p \lor q\right) \right)\)
\begin{align*}
    \left(\neg s\right) &\rightarrow \left(\left(\neg t\right) \land \neg \left(p \lor q\right) \right) &\\
    \neg s &\rightarrow \left(\left(\neg t\right) \land \neg \left(p \lor q\right) \right) & \text{Eliminamos el paréntesis del operador \(\neg\)} \\
    \neg s &\rightarrow \left(\neg t \land \neg \left(p \lor q\right) \right) & \text{Eliminamos el paréntesis del operador \(\neg\)} \\
    \neg s &\rightarrow \neg t \land \neg \left(p \lor q\right) & \text{Eliminamos el paréntesis del operador \(\land\)} \\
\end{align*}
\end{document}