\documentclass[a4paper]{article} 
\addtolength{\hoffset}{-2.25cm}
\addtolength{\textwidth}{4.5cm}
\addtolength{\voffset}{-3.25cm}
\addtolength{\textheight}{5cm}
\setlength{\parskip}{0pt}
\setlength{\parindent}{0in}

%----------------------------------------------------------------------------------------
%	PACKAGES AND OTHER DOCUMENT CONFIGURATIONS
%----------------------------------------------------------------------------------------

\usepackage{blindtext} % Package to generate dummy text
\usepackage{charter} % Use the Charter font
\usepackage[utf8]{inputenc} % Use UTF-8 encoding
\usepackage{microtype} % Slightly tweak font spacing for aesthetics
\usepackage[english, ngerman]{babel} % Language hyphenation and typographical rules
\usepackage{amsthm, amsmath, amssymb} % Mathematical typesetting
\usepackage{float} % Improved interface for floating objects
\usepackage[final, colorlinks = true, 
            linkcolor = black, 
            citecolor = black]{hyperref} % For hyperlinks in the PDF
\usepackage{graphicx, multicol} % Enhanced support for graphics
\usepackage{xcolor} % Driver-independent color extensions
\usepackage{marvosym, wasysym} % More symbols
\usepackage{rotating} % Rotation tools
\usepackage{censor} % Facilities for controlling restricted text
\usepackage{listings, style/lstlisting} % Environment for non-formatted code, !uses style file!
\usepackage{pseudocode} % Environment for specifying algorithms in a natural way
\usepackage{style/avm} % Environment for f-structures, !uses style file!
\usepackage{booktabs} % Enhances quality of tables
\usepackage{tikz-qtree} % Easy tree drawing tool
\tikzset{every tree node/.style={align=center,anchor=north},
         level distance=2cm} % Configuration for q-trees
\usepackage{style/btree} % Configuration for b-trees and b+-trees, !uses style file!
\usepackage[backend=biber,style=numeric,
            sorting=nyt]{biblatex} % Complete reimplementation of bibliographic facilities
\addbibresource{ecl.bib}
\usepackage{csquotes} % Context sensitive quotation facilities
\usepackage[yyyymmdd]{datetime} % Uses YEAR-MONTH-DAY format for dates
\renewcommand{\dateseparator}{-} % Sets dateseparator to '-'
\usepackage{fancyhdr} % Headers and footers
\pagestyle{fancy} % All pages have headers and footers
\fancyhead{}\renewcommand{\headrulewidth}{0pt} % Blank out the default header
\fancyfoot[L]{} % Custom footer text
\fancyfoot[C]{} % Custom footer text
\fancyfoot[R]{\thepage} % Custom footer text
\newcommand{\note}[1]{\marginpar{\scriptsize \textcolor{red}{#1}}} % Enables comments in red on margin

%----------------------------------------------------------------------------------------

\newcommand{\pow}[2]{#1^{#2}}
\newcommand{\supra}[1]{\textsuperscript{#1}}
\begin{document}

%-------------------------------
%	TITLE SECTION
%-------------------------------

\fancyhead[C]{}
\hrule \medskip % Upper rule
\begin{minipage}{0.295\textwidth} 
\raggedright
\footnotesize
Fausto David Hernández Jasso \hfill\\   
@FaustoJH \hfill\\
fausto.david.hernandez.jasso@ciencias.unam.mx
\end{minipage}
\begin{minipage}{0.4\textwidth} 
\centering 
\large 
Lógica Computacional I\\ 
\normalsize 
Introducción y Sintaxis de la Lógica de Proposiciones\\ 
\end{minipage}
\begin{minipage}{0.295\textwidth} 
\raggedleft
\today\hfill\\
\end{minipage}
\medskip\hrule 
\bigskip
\section{¿Por qué estudiar lógica computacional?}
\noindent
Porque la lógica en las ciencias de la computación, en cierto sentido es el \textbf{cálculo de la computación}
el cuál es el fundamento matemático para tratar la información y razonar acerca del comportamiento de programas.
\section{¿Por qué necesitamos la formalización del razonamiento correcto?}
\noindent
La lógica ha sido pieza clave para estructurar el pensamiento y el razonamiento:
\begin{itemize}
    \item Dar un fundamento a las matemáticas.
    \item Eliminar errores del razonamiento.
    \item Encontrar una forma eficiente para llegar a una justificación de una conclusión, dada cierta información en 
    forma de premisas.
\end{itemize}
\section{Argumentos lógicos}
\subsection{Definición}
\noindent
Es una colección finita de afirmaciones \textit{(proposiciones)} dividida en premisas y conclusión.
\subsection{Acerca de las premisas y de la conclusión}
\noindent
Las premisas y la conclusión debe ser susceptibles de recibir un valor de verdad. El argumento lógico puede ser
correcto o incorrecto.
\subsection{Validez de un argumento}
\noindent
Un argumento es correcto o válido si suponiendo que sus premisas son ciertas, entonces necesariamente la 
conclusión también lo es.
\subsection{Ejemplo}
\subsubsection{La isla de los caballeros y los bribones}
\noindent
En la isla de los caballeros y bribones sólo hay dos clases de habitantes, los caballeros que siempre dicen 
la verdad y los bribones que siempre mienten.
\newline
Un náufrago llega a la isla y encuentra dos habitantes: A y B. El habitante A afirma: Yo soy un bribón o 
B es un caballero. El acertijo consiste en averiguar cómo son A y B.
\subsubsection{Lógica}
\noindent
\begin{align*}
    p &:= \text{A es bribón} \\
    q &:= \text{B es un caballero} \\
    s & := p \lor q 
\end{align*}
\(s\) es lo que dijo A.
\newline
Supongamos que \(\mathcal{I}\left(p\right) = \top\)
\newline 
Así \(\mathcal{I}\left(s\right) = \bot\)
\newline 
Entonces \(\mathcal{I}\left(\neg s\right) = \top\)
\newline
Como \(\neg s \equiv \neg p \land \neg q\)
\newline 
Consecuentemente \(\mathcal{I}\left(\neg p \land \neg q\right) = \top\)
\newline 
Lo que implica que se cumpla
\begin{align*}
    \mathcal{I}\left(\neg p\right) &= \top \\
    \mathcal{I}\left(\neg q\right) &= \top \\
\end{align*}
Por lo tanto concluimos que A:
\[
    \mathcal{I}\left(p\right) = \bot
\]
Por lo tanto hemos llegado a una contradicción, ya que A no puede ser bribón y no serlo al mismo tiempo.
\newline 
Supongamos que \(\mathcal{I}\left(p\right) = \bot\)
\newline 
Así \(\mathcal{I}\left(s\right) = \top\)
\newline
Como \( s \equiv p \lor q\)
\newline 
Así tenemos que \(\mathcal{I}\left(p\right) = \bot\) y \(\mathcal{I}\left(q\right) = \top\)
\newline 
En éste caso A no es bribón y B es un caballero, entonces concluimos que:
\begin{itemize}
    \item A es un caballero.
    \item B es un caballero.
\end{itemize} 
\subsection{Características de un argumento lógico}
\begin{itemize}
    \item Involucran individuos.
    \item Los individuos que involucran tienen propiedades.
    \item Una proposición es una oración que puede calificarse como verdadera o falsa y habla de las propiedades de los individuos.
    \item Se forman medinate proposiciones, clasificadas como premisas y conclusión del argumento.
    \item Las porposiciones pueden ser compuestas.
    \item Puede ser correcto o incorrecto.
\end{itemize}
\section{Sistema lógico}
\noindent
Cualquier sistema lógico consta al menos de los siguientes tres componentes:
\begin{itemize}
    \item Sintaxis.
    \item Semántica.
    \item Teoría de la prueba.
\end{itemize}
\subsection{Sintaxis}
\noindent
Lenguaje formal que se utilizará como medio de expresión.
\subsection{Semántica}
\noindent
Mecanismo que proporciona significado al lenguaje formal dado por la sintaxis.
\subsection{Teoría de la prueba}
\noindent
Se encarga de decidir la correctud de un argumento lógico por medios puramente sintácticos.
\subsection{Propiedades}
\subsubsection{Consistencia}
\noindent
No hay contradicciones
\subsubsection{Correctud}
\noindent
Las reglas del sistema no pueden obtener una inferencia falsa a partir de una verdadera.
\subsubsection{Completud}
\noindent
Todo lo verdadero es demostrable.
\section{Lógica proposicional}
\subsection{Proposición}
\noindent
Es un enunciado que puede calificarse como verdadero o falso.
\subsection{Lenguaje \texttt{PROP}}
\begin{itemize}
    \item Símbolos o variables proposicionales \textit{(número infinito)}: \(p_{1}, \dotsc, p_{n}, \dotsc\)
    \item Constantes lógicas: \(\top, \bot\)
    \item Operadores lógicos: \(\neg, \land, \lor, \rightarrow, \leftrightarrow \)
    \item Símbolos auxiliares: \(\left(\right)\)
\end{itemize}
\end{document}