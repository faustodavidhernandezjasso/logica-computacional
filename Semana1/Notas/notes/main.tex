\documentclass[a4paper]{article} 
\addtolength{\hoffset}{-2.25cm}
\addtolength{\textwidth}{4.5cm}
\addtolength{\voffset}{-3.25cm}
\addtolength{\textheight}{5cm}
\setlength{\parskip}{0pt}
\setlength{\parindent}{0in}

%----------------------------------------------------------------------------------------
%	PACKAGES AND OTHER DOCUMENT CONFIGURATIONS
%----------------------------------------------------------------------------------------

\usepackage{blindtext} % Package to generate dummy text
\usepackage{charter} % Use the Charter font
\usepackage[utf8]{inputenc} % Use UTF-8 encoding
\usepackage{microtype} % Slightly tweak font spacing for aesthetics
\usepackage[english, ngerman]{babel} % Language hyphenation and typographical rules
\usepackage{amsthm, amsmath, amssymb} % Mathematical typesetting
\usepackage{float} % Improved interface for floating objects
\usepackage[final, colorlinks = true, 
            linkcolor = black, 
            citecolor = black]{hyperref} % For hyperlinks in the PDF
\usepackage{graphicx, multicol} % Enhanced support for graphics
\usepackage{xcolor} % Driver-independent color extensions
\usepackage{marvosym, wasysym} % More symbols
\usepackage{rotating} % Rotation tools
\usepackage{censor} % Facilities for controlling restricted text
\usepackage{listings, style/lstlisting} % Environment for non-formatted code, !uses style file!
\usepackage{pseudocode} % Environment for specifying algorithms in a natural way
\usepackage{style/avm} % Environment for f-structures, !uses style file!
\usepackage{booktabs} % Enhances quality of tables
\usepackage{tikz-qtree} % Easy tree drawing tool
\tikzset{every tree node/.style={align=center,anchor=north},
         level distance=2cm} % Configuration for q-trees
\usepackage{style/btree} % Configuration for b-trees and b+-trees, !uses style file!
\usepackage[backend=biber,style=numeric,
            sorting=nyt]{biblatex} % Complete reimplementation of bibliographic facilities
\addbibresource{ecl.bib}
\usepackage{csquotes} % Context sensitive quotation facilities
\usepackage[yyyymmdd]{datetime} % Uses YEAR-MONTH-DAY format for dates
\renewcommand{\dateseparator}{-} % Sets dateseparator to '-'
\usepackage{fancyhdr} % Headers and footers
\pagestyle{fancy} % All pages have headers and footers
\fancyhead{}\renewcommand{\headrulewidth}{0pt} % Blank out the default header
\fancyfoot[L]{} % Custom footer text
\fancyfoot[C]{} % Custom footer text
\fancyfoot[R]{\thepage} % Custom footer text
\newcommand{\note}[1]{\marginpar{\scriptsize \textcolor{red}{#1}}} % Enables comments in red on margin

%----------------------------------------------------------------------------------------

\newcommand{\pow}[2]{#1^{#2}}
\newcommand{\supra}[1]{\textsuperscript{#1}}
\begin{document}

%-------------------------------
%	TITLE SECTION
%-------------------------------

\fancyhead[C]{}
\hrule \medskip % Upper rule
\begin{minipage}{0.295\textwidth} 
\raggedright
\footnotesize
Fausto David Hernández Jasso \hfill\\   
@FaustoJH \hfill\\
fausto.david.hernandez.jasso@ciencias.unam.mx
\end{minipage}
\begin{minipage}{0.4\textwidth} 
\centering 
\large 
Lógica Computacional I\\ 
\normalsize 
Introducción y Sintaxis de la Lógica de Proposiciones\\ 
\end{minipage}
\begin{minipage}{0.295\textwidth} 
\raggedleft
\today\hfill\\
\end{minipage}
\medskip\hrule 
\bigskip
\section{¿Por qué estudiar lógica computacional?}
\noindent
Porque la lógica en las ciencias de la computación, en cierto sentido es el \textbf{cálculo de la computación}
el cuál es el fundamento matemático para tratar la información y razonar acerca del comportamiento de programas.
\section{¿Por qué necesitamos la formalización del razonamiento correcto?}
\noindent
La lógica ha sido pieza clave para estructurar el pensamiento y el razonamiento:
\begin{itemize}
    \item Dar un fundamento a las matemáticas.
    \item Eliminar errores del razonamiento.
    \item Encontrar una forma eficiente para llegar a una justificación de una conclusión, dada cierta información en 
    forma de premisas.
\end{itemize}
\section{Argumentos lógicos}
\subsection{Definición}
\noindent
Es una colección finita de afirmaciones \textit{(proposiciones)} dividida en premisas y conclusión.
\subsection{Acerca de las premisas y de la conclusión}
\noindent
Las premisas y la conclusión debe ser susceptibles de recibir un valor de verdad. El argumento lógico puede ser
correcto o incorrecto.
\subsection{Validez de un argumento}
\noindent
Un argumento es correcto o válido si suponiendo que sus premisas son ciertas, entonces necesariamente la 
conclusión también lo es.
\subsection{Ejemplo}
\subsubsection{La isla de los caballeros y los bribones}
\noindent
En la isla de los caballeros y bribones sólo hay dos clases de habitantes, los caballeros que siempre dicen 
la verdad y los bribones que siempre mienten.
\newline
Un náufrago llega a la isla y encuentra dos habitantes: A y B. El habitante A afirma: Yo soy un bribón o 
B es un caballero. El acertijo consiste en averiguar cómo son A y B.
\subsubsection{Lógica}
\noindent
\begin{align*}
    p &:= \text{A es bribón} \\
    q &:= \text{B es un caballero} \\
    s & := p \lor q 
\end{align*}
\(s\) es lo que dijo A.
\newline
Supongamos que \(\mathcal{I}\left(p\right) = \top\)
\newline 
Así \(\mathcal{I}\left(s\right) = \bot\)
\newline 
Entonces \(\mathcal{I}\left(\neg s\right) = \top\)
\newline
Como \(\neg s \equiv \neg p \land \neg q\)
\newline 
Consecuentemente \(\mathcal{I}\left(\neg p \land \neg q\right) = \top\)
\newline 
Lo que implica que se cumpla
\begin{align*}
    \mathcal{I}\left(\neg p\right) &= \top \\
    \mathcal{I}\left(\neg q\right) &= \top \\
\end{align*}
Por lo tanto concluimos que A:
\[
    \mathcal{I}\left(p\right) = \bot
\]
Por lo tanto hemos llegado a una contradicción, ya que A no puede ser bribón y no serlo al mismo tiempo.
\newline 
Supongamos que \(\mathcal{I}\left(p\right) = \bot\)
\newline 
Así \(\mathcal{I}\left(s\right) = \top\)
\newline
Como \( s \equiv p \lor q\)
\newline 
Así tenemos que \(\mathcal{I}\left(p\right) = \bot\) y \(\mathcal{I}\left(q\right) = \top\)
\newline 
En éste caso A no es bribón y B es un caballero, entonces concluimos que:
\begin{itemize}
    \item A es un caballero.
    \item B es un caballero.
\end{itemize} 
\subsection{Características de un argumento lógico}
\begin{itemize}
    \item Involucran individuos.
    \item Los individuos que involucran tienen propiedades.
    \item Una proposición es una oración que puede calificarse como verdadera o falsa y habla de las propiedades de los individuos.
    \item Se forman medinate proposiciones, clasificadas como premisas y conclusión del argumento.
    \item Las porposiciones pueden ser compuestas.
    \item Puede ser correcto o incorrecto.
\end{itemize}
\section{Sistema lógico}
\noindent
Cualquier sistema lógico consta al menos de los siguientes tres componentes:
\begin{itemize}
    \item Sintaxis.
    \item Semántica.
    \item Teoría de la prueba.
\end{itemize}
\subsection{Sintaxis}
\noindent
Lenguaje formal que se utilizará como medio de expresión.
\subsection{Semántica}
\noindent
Mecanismo que proporciona significado al lenguaje formal dado por la sintaxis.
\subsection{Teoría de la prueba}
\noindent
Se encarga de decidir la correctud de un argumento lógico por medios puramente sintácticos.
\subsection{Propiedades}
\subsubsection{Consistencia}
\noindent
No hay contradicciones
\subsubsection{Correctud}
\noindent
Las reglas del sistema no pueden obtener una inferencia falsa a partir de una verdadera.
\subsubsection{Completud}
\noindent
Todo lo verdadero es demostrable.
\section{Lógica proposicional}
\subsection{Proposición}
\noindent
Es un enunciado que puede calificarse como verdadero o falso.
\subsection{Lenguaje \texttt{PROP}}
\begin{itemize}
    \item Símbolos o variables proposicionales \textit{(número infinito)}: \(p_{1}, \dotsc, p_{n}, \dotsc\)
    \item Constantes lógicas: \(\top, \bot\)
    \item Operadores lógicos: \(\neg, \land, \lor, \rightarrow, \leftrightarrow \)
    \item Símbolos auxiliares: \(\left(\right)\)
\end{itemize}
El conjunto de expresiones o fórmulas atómicas, denotado como \texttt{ATOM} consta de:
\begin{itemize}
    \item Variables proposicionales: \(p_{1}, p_{2}, \dotsc, p_{n}, \dotsc\)
    \item Constantes lógicas: \(\top, \bot\)
\end{itemize}
\subsection{Expresiones en el lenguaje \texttt{PROP}}
\subsubsection{Definición recursiva}
\begin{itemize}
    \item Si \(\varphi \in \text{\texttt{ATOM}}\) entonces \(\varphi \in \text{\texttt{PROP}}\).
    \item Si \(\varphi \in \text{\texttt{PROP}}\) entonces \(\left(\neg \varphi\right) \in \text{\texttt{PROP}}\).
    \item Sí \(\varphi, \psi \in \text{\texttt{PROP}}\) entonces \(\left(\varphi \land \psi \right), \left(\varphi \lor \psi \right), \left(\varphi \rightarrow \psi \right), \left(\varphi \leftrightarrow \psi \right) \in \text{\texttt{PROP}}\).
    \item Son todas.
\end{itemize}
\subsubsection{Backus-Naur}
\begin{align*}
    VarP &::= p_{1} \ | \ p_{2} \ | \ \dotsc \ | \ p_{n} \ | \ \dotsc  \\
    \varphi, \psi &::= VarP \ | \ \top \ | \ \bot \ | \ \left(\neg \varphi\right)  \ | \ \left(\varphi \land \psi \right) \ | \ \left(\varphi \lor \psi \right)  \ | \ \left(\varphi \rightarrow \psi \right)  \ | \ \left(\varphi \leftrightarrow \psi \right)  \ |
\end{align*}
\subsubsection{¿Cómo construimos proposiciones a través de la gramática anterior?}
\noindent
Veamos cómo construir \(\left(p \land q\right) \rightarrow \left(\left(\neg q \right) \lor s\right)\)
\begin{align*}
    \varphi &\rightarrow \psi \\
    \left(\varphi_{1} \land \psi_{1} \right) &\rightarrow \psi \\
    \left(\varphi_{1} \land \psi_{1} \right) &\rightarrow \left(\varphi_{2} \lor \psi_{2} \right) \\
    \left(\varphi_{1} \land \psi_{1} \right) &\rightarrow \left(\left(\neg \varphi_{2}\right) \lor \psi_{2} \right) \\
    \left(VarP_{\varphi_{1}} \land VarP_{\psi_{1}} \right) &\rightarrow \left(\left(\neg VarP_{\varphi_{2}}\right) \lor VarP_{\psi_{2}} \right) \\
    \left(p \land VarP_{\psi_{1}} \right) &\rightarrow \left(\left(\neg VarP_{\varphi_{2}}\right) \lor VarP_{\psi_{2}} \right) \\
    \left(p \land q \right) &\rightarrow \left(\left(\neg VarP_{\varphi_{2}}\right) \lor VarP_{\psi_{2}} \right) \\
    \left(p \land q \right) &\rightarrow \left(\left(\neg q\right) \lor VarP_{\psi_{2}} \right) \\
    \left(p \land q \right) &\rightarrow \left(\left(\neg q\right) \lor s \right)
\end{align*}
Veamos cómo construir \(\left(\left(p \lor q\right) \rightarrow r\right) \leftrightarrow \left(\left(\neg r\right) \rightarrow \left(\neg\left(p \lor q\right)\right) \right)\)
\begin{align*}
    \varphi &\leftrightarrow \psi \\
    \left(\varphi_{1} \rightarrow \psi_{1} \right) &\leftrightarrow \psi \\
    \left(\varphi_{1} \rightarrow \psi_{1} \right) &\leftrightarrow \left(\varphi_{2} \rightarrow \psi_{2} \right) \\
    \left(\left(\varphi_{3} \lor \psi_{3} \right) \rightarrow \psi_{1} \right) &\leftrightarrow \left(\varphi_{2} \rightarrow \psi_{2} \right) \\
    \left(\left(\varphi_{3} \lor \psi_{3} \right) \rightarrow \psi_{1} \right) &\leftrightarrow \left(\left(\neg \varphi_{2}\right) \rightarrow \psi_{2} \right) \\
    \left(\left(\varphi_{3} \lor \psi_{3} \right) \rightarrow \psi_{1} \right) &\leftrightarrow \left(\left(\neg \varphi_{2}\right) \rightarrow \left(\neg \psi_{2}\right) \right) \\
    \left(\left(\varphi_{3} \lor \psi_{3} \right) \rightarrow \psi_{1} \right) &\leftrightarrow \left(\left(\neg \varphi_{2}\right) \rightarrow \left(\neg \left(\varphi_{4} \lor \psi_{4}\right)\right) \right) \\
    \left(\left(VarP_{\varphi_{3}} \lor VarP_{\psi_{3}} \right) \rightarrow VarP_{\psi_{1}} \right) &\leftrightarrow \left(\left(\neg VarP_{\varphi_{2}}\right) \rightarrow \left(\neg \left(VarP_{\varphi_{4}} \lor VarP_{\psi_{4}}\right)\right) \right) \\
    \left(\left(p \lor VarP_{\psi_{3}} \right) \rightarrow VarP_{\psi_{1}} \right) &\leftrightarrow \left(\left(\neg VarP_{\varphi_{2}}\right) \rightarrow \left(\neg \left(VarP_{\varphi_{4}} \lor VarP_{\psi_{4}}\right)\right) \right) \\
    \left(\left(p \lor q \right) \rightarrow VarP_{\psi_{1}} \right) &\leftrightarrow \left(\left(\neg VarP_{\varphi_{2}}\right) \rightarrow \left(\neg \left(VarP_{\varphi_{4}} \lor VarP_{\psi_{4}}\right)\right) \right) \\
    \left(\left(p \lor q \right) \rightarrow r \right) &\leftrightarrow \left(\left(\neg VarP_{\varphi_{2}}\right) \rightarrow \left(\neg \left(VarP_{\varphi_{4}} \lor VarP_{\psi_{4}}\right)\right) \right) \\
    \left(\left(p \lor q \right) \rightarrow r \right) &\leftrightarrow \left(\left(\neg r\right) \rightarrow \left(\neg \left(VarP_{\varphi_{4}} \lor VarP_{\psi_{4}}\right)\right) \right) \\
    \left(\left(p \lor q \right) \rightarrow r \right) &\leftrightarrow \left(\left(\neg r\right) \rightarrow \left(\neg \left(p \lor VarP_{\psi_{4}}\right)\right) \right) \\
    \left(\left(p \lor q \right) \rightarrow r \right) &\leftrightarrow \left(\left(\neg r\right) \rightarrow \left(\neg \left(p \lor q\right)\right) \right)
\end{align*}
Veamos cómo construir \(\neg\left(\left(\left(\left(p \land \left(p \lor \left(\neg q\right)\right)\right) \land q\right) \land q\right)\right)\)
\begin{align*}
    \neg&\left(\varphi\right) \\
    \neg&\left(\left(\varphi_{1} \land \psi_{1}\right)\right) \\
    \neg&\left(\left(\left(\varphi_{2} \land \psi_{2}\right) \land \psi_{1}\right)\right) \\
    \neg&\left(\left(\left(\left(\varphi_{3} \land \psi_{3}\right) \land \psi_{2}\right) \land \psi_{1}\right)\right) \\
    \neg&\left(\left(\left(\left(\varphi_{3} \land \left(\varphi_{4} \lor \psi_{4}\right)\right) \land \psi_{2}\right) \land \psi_{1}\right)\right) \\
    \neg&\left(\left(\left(\left(\varphi_{3} \land \left(\varphi_{4} \lor \left(\neg \psi_{4}\right)\right)\right) \land \psi_{2}\right) \land \psi_{1}\right)\right) \\
    \neg&\left(\left(\left(\left(VarP_{\varphi_{3}} \land \left(VarP_{\varphi_{4}} \lor \left(\neg VarP_{\psi_{4}}\right)\right)\right) \land VarP_{\psi_{2}}\right) \land VarP_{\psi_{1}}\right)\right) \\
    \neg&\left(\left(\left(\left(p \land \left(VarP_{\varphi_{4}} \lor \left(\neg VarP_{\psi_{4}}\right)\right)\right) \land VarP_{\psi_{2}}\right) \land VarP_{\psi_{1}}\right)\right) \\
    \neg&\left(\left(\left(\left(p \land \left(p \lor \left(\neg VarP_{\psi_{4}}\right)\right)\right) \land VarP_{\psi_{2}}\right) \land VarP_{\psi_{1}}\right)\right) \\
    \neg&\left(\left(\left(\left(p \land \left(p \lor \left(\neg q\right)\right)\right) \land VarP_{\psi_{2}}\right) \land VarP_{\psi_{1}}\right)\right) \\
    \neg&\left(\left(\left(\left(p \land \left(p \lor \left(\neg q\right)\right)\right) \land q\right) \land VarP_{\psi_{1}}\right)\right) \\
    \neg&\left(\left(\left(\left(p \land \left(p \lor \left(\neg q\right)\right)\right) \land q\right) \land q\right)\right)
\end{align*}
Veamos cómo construir \(\left(\neg s\right) \rightarrow \left(\left(\neg t\right) \land \neg \left(p \lor q\right) \right)\)
\begin{align*}
    (\varphi & \rightarrow \psi) \\
    ((\neg \varphi) & \rightarrow \psi) \\
    ((\neg \varphi) &\rightarrow (\varphi_{1} \land \psi_{1})) \\
    ((\neg \varphi) &\rightarrow ((\neg \varphi_{1}) \land \psi_{1})) \\
    ((\neg \varphi) &\rightarrow ((\neg \varphi_{1}) \land (\neg \psi_{1}))) \\
    ((\neg \varphi) &\rightarrow ((\neg \varphi_{1}) \land (\neg (\varphi_{2} \lor \psi_{2})))) \\
    ((\neg VarP_{\varphi}) &\rightarrow ((\neg VarP_{\varphi_{1}}) \land (\neg (VarP_{\varphi_{2}} \lor VarP_{\psi_{2}})))) \\
    ((\neg s) &\rightarrow ((\neg VarP_{\varphi_{1}}) \land (\neg (VarP_{\varphi_{2}} \lor VarP_{\psi_{2}})))) \\
    ((\neg s) &\rightarrow ((\neg t) \land (\neg (VarP_{\varphi_{2}} \lor VarP_{\psi_{2}})))) \\
    ((\neg s) &\rightarrow ((\neg t) \land (p \lor q)))) \\
\end{align*}
\subsubsection{Ejercicios}
Realizar las siguientes construcciones:
\begin{itemize}
    \item \(\left(s \lor r\right) \land \left(p \rightarrow \left(p \rightarrow \neg \left(q \land \left(s \lor q\right)\right)\right)\right)\)
    \item \(\left(\left(\neg \left(r \rightarrow t\right)\right) \leftrightarrow \left(\left(t \land \left(p \rightarrow \left(q \lor t\right)\right)\right) \leftrightarrow s \right)\right)\)
    \item \(\left(\left(p \rightarrow \left(q \leftrightarrow \left(s \land \left(t \rightarrow q\right)\right)\right)\right)\land \left(q \lor \left(s \lor \left(t \leftrightarrow \left(q \lor \left(\neg r\right)\right)\right)\right)\right) \right)\)
\end{itemize}
% \begin{align*}
%     VarP_{}
%     \left(\varphi &\rightarrow \psi \right) \\ 
%     \left(\left(\neg \varphi\right) & \rightarrow \psi \right) \\
% \end{align*}
\subsection{Precedencia y asociatividad de los operadores lógicos}
Precedencia de mayor a menor
\begin{itemize}
    \item \(\neg\)
    \item \(\lor, \land\)
    \item \(\rightarrow\)
    \item \(\leftrightarrow\)
\end{itemize}
Operadores asociativos:
\begin{itemize}
    \item \(\neg\)
    \item \(\lor, \land\)
    \item \(\leftrightarrow\)
\end{itemize}
El operador \(\rightarrow\) asocia hacia la derecha.
\newline 
Ejemplificando lo anterior la expresión
\[
    \varphi_{1} \rightarrow \varphi_{2} \rightarrow \varphi_{3}  
\]
Queda asociada como:
\[
    \varphi_{1} \rightarrow \left(\varphi_{2} \rightarrow \varphi_{3}\right)  
\]
\subsection{Eliminación de paréntesis innecesarios}
\noindent
Notemos que:
\[
     p \land q \rightarrow \neg q \lor s  
\]
es igual a
\[
    \left(p \land q\right) \rightarrow \left(\left(\neg q \right) \lor s\right)  
\]
Pero ¿Cómo podemos obtener la primera sí nos dan la última?
\newline 
Simplemente debemos de ir quitando paréntesis guiándonos por la precedencia de operadores
\begin{align*}
    \left(p \land q\right) &\rightarrow \left(\left(\neg q \right) \lor s\right) &\\
    \left(p \land q\right) &\rightarrow \left(\neg q \lor s\right) &\text{Eliminamos el paréntesis del operador \(\neg\)}\\
    p \land q &\rightarrow \neg q \lor s &\text{Eliminamos el paréntesis del operador \(\lor\) y \(\land\)}
\end{align*}
Veamos \(\left(p \rightarrow q \land r\right) \leftrightarrow \left(s \lor \left(\neg t\right)\right)\)
\begin{align*}
    \left(p \rightarrow q \land r\right) &\leftrightarrow \left(s \lor \left(\neg t\right)\right) &\\
    \left(p \rightarrow q \land r\right) &\leftrightarrow \left(s \lor \neg t\right) &\text{Eliminamos el paréntesis del operador \(\neg\)}\\
    \left(p \rightarrow q \land r\right) &\leftrightarrow s \lor \neg t &\text{Eliminamos el paréntesis del operador \(\lor\)}\\
    p \rightarrow q \land r &\leftrightarrow s \lor \neg t &\text{Eliminamos el paréntesis del operador \(\rightarrow\)}\\
\end{align*}
Veamos \(\left(\left(p \lor q\right) \rightarrow r\right) \leftrightarrow \left(\left(\neg r\right) \rightarrow \left(\neg\left(p \lor q\right)\right) \right)\)
\begin{align*}
    \left(\left(p \lor q\right) \rightarrow r\right) &\leftrightarrow \left(\left(\neg r\right) \rightarrow \left(\neg\left(p \lor q\right)\right) \right) & \\
    \left(\left(p \lor q\right) \rightarrow r\right) &\leftrightarrow \left(\neg r \rightarrow \left(\neg\left(p \lor q\right)\right) \right) & \text{Eliminamos el paréntesis del operador \(\neg\)} \\
    \left(\left(p \lor q\right) \rightarrow r\right) &\leftrightarrow \left(\neg r \rightarrow \neg\left(p \lor q\right) \right) & \text{Eliminamos el paréntesis del operador \(\neg\)} \\
    p \lor q \rightarrow r &\leftrightarrow \left(\neg r \rightarrow \neg\left(p \lor q\right) \right) & \text{Eliminamos el paréntesis del operador \(\rightarrow\)} \\
    p \lor q \rightarrow r &\leftrightarrow \neg r \rightarrow \neg\left(p \lor q\right) & \text{Eliminamos el paréntesis del operador \(\rightarrow\)}
\end{align*}
Veamos \(\neg\left(\left(\left(p \land \left(p \lor \left(\neg q\right)\right)\right) \land q \right) \land p\right)\)
\begin{align*}
    &\neg\left(\left(\left(p \land \left(p \lor \left(\neg q\right)\right)\right) \land q \right) \land p\right) & \\
    &\neg\left(\left(\left(p \land \left(p \lor \neg q\right)\right) \land q \right) \land p\right) & \text{Eliminamos el paréntesis del operador \(\neg\)}\\
    &\neg\left(\left(p \land \left(p \lor \neg q\right) \land q \right) \land p\right) & \text{Eliminamos el paréntesis del operador \(\land\)}\\
    &\neg\left(p \land \left(p \lor \neg q\right) \land q \land p\right) & \text{Eliminamos el paréntesis del operador \(\land\)}\\
\end{align*}
Veamos \(\left(\neg s\right) \rightarrow \left(\left(\neg t\right) \land \neg \left(p \lor q\right) \right)\)
\begin{align*}
    \left(\neg s\right) &\rightarrow \left(\left(\neg t\right) \land \neg \left(p \lor q\right) \right) &\\
    \neg s &\rightarrow \left(\left(\neg t\right) \land \neg \left(p \lor q\right) \right) & \text{Eliminamos el paréntesis del operador \(\neg\)} \\
    \neg s &\rightarrow \left(\neg t \land \neg \left(p \lor q\right) \right) & \text{Eliminamos el paréntesis del operador \(\neg\)} \\
    \neg s &\rightarrow \neg t \land \neg \left(p \lor q\right) & \text{Eliminamos el paréntesis del operador \(\land\)} \\
\end{align*}
\subsubsection{Ejercicios}
\noindent
Elimina los paréntesis innecesarios de la expresiones:
\begin{itemize}
    \item \(\left(p \rightarrow \left(q \land \left(\neg q\right)\right)\right) \rightarrow \left(\left(\neg p\right) \rightarrow p\right)\)
    \item \(\left(s \lor r\right) \land \left(p \rightarrow \left(p \rightarrow \neg \left(q \land \left(s \lor q\right)\right)\right)\right)\)
\end{itemize}
\section{Definiciones Recursivas y el Principio de Inducción}
\subsection{¿En qué consiste?}
\noindent
En establecer las construcciones para generar elementos de una colección o de una estructura de datos.
\newline
La definición del lenguaje \texttt{PROP} es un ejemplo de ésta clase de definiciones, podemos identificar las 
construcciones básicas o atómicas y las construcciones recursivas.
\newline 
Las definiciones recursivas también sirven para definir propiedades o funciones de una estructura mediante un
análisis de casos, es decir de las distintas formas sintácticas que definen a los elementos de dicha estructura.
\subsection{Ejemplo}
\noindent
Sea \(np\) una función que devuelve el número de paréntesis de una expresión lógica.
\newline 
Definimos \(np\) como sigue:
\begin{align*}
    np \ &: \ \text{\texttt{PROP}} \to \mathbb{N} &  \\ 
    np\left(\varphi\right) &= 0 & \text{Si \(\varphi \in \text{\texttt{ATOM}}\)} \\ 
    np\left(\left(\neg \varphi \right)\right) &= 2 + np\left(\varphi\right) & \text{Si \(\varphi \in \text{\texttt{ATOM}}\)} \\ 
    np\left(\left(\varphi \star \psi \right)\right) &= 2 + np\left(\varphi\right) + np\left(\psi\right) & \text{con \(\star \in \left\{\land, \lor, \rightarrow, \leftrightarrow\right\}\)} \\ 
    np\left(\neg \varphi\right) &= np\left(\varphi\right) \\
    np\left(\varphi \star \psi \right) &= np\left(\varphi\right) + np\left(\psi\right) & \text{con \(\star \in \left\{\land, \lor, \rightarrow, \leftrightarrow\right\}\)}
\end{align*}
\subsection{Ejemplos de uso}
\noindent
Calculemos \(np\left(\neg\left(\left(\left(p \land \left(p \lor \left(\neg q\right)\right)\right) \land q \right) \land p\right)\right)\)
\begin{align*}
    np\left(\neg\left(\left(\left(p \land \left(p \lor \left(\neg q\right)\right)\right) \land q \right) \land p\right)\right) &= 2 + np\left(\left(\left(p \land \left(p \lor \left(\neg q\right)\right)\right) \land q \right) \land p\right)  &\text{Aplicamos el segundo caso.} \\
    np\left(\left(\left(p \land \left(p \lor \left(\neg q\right)\right)\right) \land q \right) \land p\right) &= 2 + np\left(\left(\left(p \land \left(p \lor \left(\neg q\right)\right)\right) \land q \right)\right) + np\left(p\right)  &\text{Aplicamos el quinto caso.} \\
    np\left(\left(\left(p \land \left(p \lor \left(\neg q\right)\right)\right) \land q \right)\right) &= 2 + np\left(\left(p \land \left(p \lor \left(\neg q\right)\right)\right) \right) +  np\left(q\right)  &\text{Aplicamos el tercer caso.} \\
    np\left(\left(p \land \left(p \lor \left(\neg q\right)\right)\right) \right) &= 2 + np\left(p\right) + np\left(\left(p \lor \left(\neg q\right)\right)\right)  &\text{Aplicamos el tercer caso.} \\
    np\left(p\right) &= 0  &\text{Aplicamos el primer caso.} \\
    np\left(\left(p \lor \left(\neg q\right)\right)\right) &= 2 + np\left(p\right) + np\left(\left(\neg q\right)\right)  &\text{Aplicamos el tercer caso.} \\
    np\left(p\right) &= 0  &\text{Aplicamos el primer caso.} \\
    np\left(\left(\neg q\right)\right) &= 2 + np\left(q\right) & \text{Aplicamos el segundo caso.} \\
    np\left(q\right) &= 0 & \text{Aplicamos el primer caso.} \\
\end{align*}
Por lo tanto
\begin{align*}
    np\left(\left(\neg q\right)\right) &= 2 + np\left(q\right) = 2 + 0 = 2 \\
    np\left(\left(p \lor \left(\neg q\right)\right)\right) &= 2 + np\left(p\right) + np\left(\left(\neg q\right)\right) = 2 + 0 + 2 = 4 \\
    np\left(\left(p \land \left(p \lor \left(\neg q\right)\right)\right) \right) &= 2 + np\left(p\right) + np\left(\left(p \lor \left(\neg q\right)\right)\right) = 2 + 0 + 4 = 6 \\ 
    np\left(\left(\left(p \land \left(p \lor \left(\neg q\right)\right)\right) \land q \right)\right) &= 2 + np\left(\left(p \land \left(p \lor \left(\neg q\right)\right)\right) \right) +  np\left(q\right) = 2 + 6 + 0 = 8 \\
    np\left(\left(\left(p \land \left(p \lor \left(\neg q\right)\right)\right) \land q \right) \land p\right) &= 2 + np\left(\left(\left(p \land \left(p \lor \left(\neg q\right)\right)\right) \land q \right)\right) + np\left(p\right) = 2 + 8 + 0 = 10\\
    np\left(\neg\left(\left(\left(p \land \left(p \lor \left(\neg q\right)\right)\right) \land q \right) \land p\right)\right) &= 2 + np\left(\left(\left(p \land \left(p \lor \left(\neg q\right)\right)\right) \land q \right) \land p\right) = 2 + 10 = 12
\end{align*}
Por lo tanto el número de paréntesis que tiene la fórmula son 12.
\subsection{Definición de funciones recursivas}
\noindent
Define las siguientes funciones, indicando dominio y contradominio de la función definida.
\begin{itemize}
    \item Número de conectivos de una expresión: \(con\left(\varphi\right)\) devuelve el número de conectivos de \(\varphi\).
    \item Variables de una expresión: \(vars\left(\varphi\right)\) devuelve el conjunto de variables que figuran en \(\varphi\).
    \item Atómicas en una expresión: \(atom\left(\varphi\right)\) devuelve el número de presencias de fórmulas atómicas en \(\varphi\).
\end{itemize}
\subsubsection{Número de conectivos}
\noindent
Recordemos que los conectivos son:
\[
    \{\neg, \land, \lor, \rightarrow, \Leftrightarrow\}  
\]
Así tenemos que la correspondencia de nuestra función es:
\[
      con \ : \ \text{\texttt{PROP}} \to \mathbb{N}
\]
Definición de la función
\begin{align*}
    con\left(\varphi\right) &= 0 &\text{Sí \(\varphi \in\) \texttt{ATOM}} \\
    con\left(\neg \varphi\right) &= 1 + con\left(\varphi\right) & \\
    con\left(\varphi \star \psi \right) &= 1 + con\left(\varphi\right) + cons\left(\psi\right) & \text{con \(\star \in \{ \neg, \land, \lor, \rightarrow, \Leftrightarrow \}\)} \\
\end{align*}
Hacemos la observación que \(\varphi\) y \(\psi\) pueden contener o no paréntesis es decir ambas pueden ser de la forma:
\[
    \left(\kappa \star \omega \right)  
\]
con \(\star \in \{ \neg, \land, \lor, \rightarrow, \Leftrightarrow \}\) ó
\[
    \left(\neg \gamma\right)
\]
Para \(\kappa, \omega, \gamma \in \text{\texttt{PROP}}\).
\subsubsection{Variables de una fórmula}
Tenemos que la correspondencia de nuestra función es:
\[
      vars \ : \ \text{\texttt{PROP}} \to \{\text{\texttt{PROP}}\}
\]
Definición de la función
\begin{align*}
    vars\left(\varphi\right) &= \{ \varphi \} &\text{Sí \(\varphi \in\) \texttt{ATOM\(\setminus \left\{\bot, \top\right\}\)}} \\
    vars\left(\neg \varphi\right) &= vars\left(\varphi\right) & \\
    vars\left(\varphi \star \psi \right) &= vars\left(\varphi\right) \cup vars\left(\psi\right) & \text{con \(\star \in \{ \neg, \land, \lor, \rightarrow, \Leftrightarrow \}\)} \\
\end{align*}
Hacemos la observación que \(\varphi\) y \(\psi\) pueden contener o no paréntesis es decir ambas pueden ser de la forma:
\[
    \left(\kappa \star \omega \right)  
\]
con \(\star \in \{ \neg, \land, \lor, \rightarrow, \Leftrightarrow \}\) ó
\[
    \left(\neg \gamma\right)
\]
Para \(\kappa, \omega, \gamma \in \text{\texttt{PROP}}\).
\subsubsection{Atómicas de una fórmula}
\noindent
Tenemos que la correspondencia de nuestra función es:
\[
      atom \ : \ \text{\texttt{PROP}} \to \mathbb{N}
\]
Definición de la función
\begin{align*}
    atom\left(\varphi\right) &= 1 &\text{Sí \(\varphi \in\) \texttt{ATOM}} \\
    atom\left(\neg \varphi\right) &= atom\left(\varphi\right) & \\
    atom\left(\varphi \star \psi \right) &= atom\left(\varphi\right) + atom\left(\psi\right) & \text{con \(\star \in \{ \neg, \land, \lor, \rightarrow, \Leftrightarrow \}\)} \\
\end{align*}
Hacemos la observación que \(\varphi\) y \(\psi\) pueden contener o no paréntesis es decir ambas pueden ser de la forma:
\[
    \left(\kappa \star \omega \right)  
\]
con \(\star \in \{ \neg, \land, \lor, \rightarrow, \Leftrightarrow \}\) ó
\[
    \left(\neg \gamma\right)
\]
Para \(\kappa, \omega, \gamma \in \text{\texttt{PROP}}\).
\subsubsection{Ejercicio}
\noindent
Define una función \(depth\left(\varphi\right)\) que devuelve la altura del árbol de análisis sintáctico de \(\varphi\).
\subsection{Principio de Inducción Estructural para \texttt{PROP}}
\noindent
Sea \(\mathcal{P}\) una propiedad acerca de fórmulas del lenguaje \texttt{PROP}. Para probar que toda fórmula 
\(\varphi \in \text{\texttt{PROP}}\) tiene la propiedad \(\mathcal{P}\) basta demostrar lo siguiente:
\begin{enumerate}
    \item \textbf{Caso base: }toda variable proposicional tiene la propiedad \(\mathcal{P}\).
    \item \textbf{Hipótesis de Inducción: } suponer que se cumple la propiedad \(\mathcal{P}\) para \(\varphi\) y \(\psi\).
    \item \textbf{Paso Inductivo: } mostrar usando la hipótesis de inducción que:
    \begin{enumerate}
        \item \(\left(\neg \varphi\right)\) cumple \(\mathcal{P}\).
        \item \(\left(\varphi \star \psi\right)\) cumple \(\mathcal{P}\) para todo \(\star \in \{\land, \lor, \rightarrow, \leftrightarrow\}\).
    \end{enumerate}
\end{enumerate}
\subsubsection{Ejemplo}
\noindent
Demostrar mediante inducción estructural que se cumplen la siguiente desigualdad
\[
    atom\left(\varphi\right)  \leq 2 \cdot con\left(\varphi\right) + 1
\]
\begin{proof}
    La demostración se hará por medio del \textbf{Principio de Inducción Estructural para \texttt{PROP}}.
    \newline
    \textbf{Caso Base}
    \newline
    Sea \(\varphi\) una variable proposicional entonces por definición de la función \(atom\) tenemos que:
    \[
        atom\left(\varphi\right) = 1
    \]
    Mientras por definición de la función \(cons\) tenemos que:
    \[
        2 \cdot con\left(\varphi\right) + 1 = 2 \cdot 0 + 1 = 0 + 1 = 1
    \]
    Entonces se cumple que
    \[
        1 \leq 1  
    \]
    Consecuentemente
    \[
        atom\left(\varphi\right) \leq  2 \cdot con\left(\varphi\right) 
    \]
    \textbf{Hipótesis de Inducción}
    \newline 
    Sean \(\varphi\) y \(\psi\) en \texttt{PROP} entonces \(\varphi\) y \(\psi\) cumplen que:
    \begin{align*}
        atom\left(\varphi\right)  &\leq 2 \cdot con\left(\varphi\right) + 1 \\
        atom\left(\psi\right)  &\leq 2 \cdot con\left(\psi\right) + 1 \\
    \end{align*}
    \textbf{Paso inductivo}
    \newline 
    Sea \(\gamma \in \text{\texttt{PROP}}\) entonces demostraremos que 
    \[
        atom\left(\gamma\right)  \leq 2 \cdot con\left(\gamma\right) + 1  
    \]
    Caso 1
    \newline 
    \(\gamma \equiv \neg \varphi\)
    \newline 
    Por definición de la función \(atom\) tenemos que:
    \[
        atom\left(\neg \varphi\right) = atom\left(\varphi\right)
    \]
    Por \textbf{hipótesis de inducción}
    \[
        atom\left(\varphi\right) \leq 2 \cdot con\left(\varphi\right) + 1
    \]
    Notemos que:
    \[
        2 \cdot con\left(\varphi\right) + 1 = con\left(\varphi\right) + con\left(\varphi\right) + 1
    \]
    Es trivial ver que:
    \[
        con\left(\varphi\right) + con\left(\varphi\right) + 1 < con\left(\varphi\right) +  con\left(\varphi\right) + 1 + 1 + 1
    \]
    Por asociatividad tenemos que:
    \[
        con\left(\varphi\right) + con\left(\varphi\right) + 1 + 1 + 1 = con\left(\varphi\right) + 1 + con\left(\varphi\right) + 1 + 1
    \]
    Que por definición de la función \(con\) es:
    \[
        2 \cdot con\left(\varphi\right) + 1 < con\left(\neg \varphi\right) + con\left(\neg \varphi\right) + 1
    \]
    Que es equivalente a 
    \[
        2 \cdot con\left(\varphi\right) + 1 < 2 \cdot con\left(\neg \varphi\right) + 1
    \]
    Así tenemos que
    \[
        atom\left(\varphi\right) \leq 2 \cdot con\left(\varphi\right) + 1 < 2 \cdot con\left(\neg \varphi\right) + 1
    \]
    Por lo que podemos concluir que
    \[
        atom\left(\neg \varphi\right) \leq 2 \cdot con\left(\neg \varphi\right) + 1
    \]
    Caso 2
    \newline 
    \(\gamma \equiv \varphi \star \psi\)
    \newline
    Por definición de la función \(atom\) tenemos que:
    \[
        atom\left(\varphi \star \psi\right) = atom\left(\varphi\right) + atom\left(\psi\right) 
    \]
    Por \textbf{hipótesis de inducción}
    \begin{align*}
        atom\left(\varphi\right)  &\leq 2 \cdot con\left(\varphi\right) + 1 \\
        atom\left(\psi\right)  &\leq 2 \cdot con\left(\psi\right) + 1 \\
    \end{align*}
    Así
    \begin{align*}
        atom\left(\varphi\right) + atom\left(\psi\right) &\leq 2 \cdot con\left(\varphi\right) + 1 + 2 \cdot con\left(\psi\right) + 1 \\
        atom\left(\varphi\right) + atom\left(\psi\right) &\leq con\left(\varphi\right) + con\left(\varphi\right) + 1 + con\left(\psi\right) + con\left(\psi\right) + 1 \\
        atom\left(\varphi\right) + atom\left(\psi\right) &\leq con\left(\varphi\right) + con\left(\psi\right) + 1 + con\left(\varphi\right) + con\left(\psi\right) + 1 \\
        atom\left(\varphi\right) + atom\left(\psi\right) &\leq con\left(\varphi \star \psi\right) + con\left(\varphi \star \psi\right) \\
        atom\left(\varphi\right) + atom\left(\psi\right) &\leq 2 \cdot con\left(\varphi \star \psi\right)
    \end{align*}
    Es inmediato ver que:
    \begin{align*}
        2 \cdot con\left(\varphi \star \psi\right) &< 2 \cdot con\left(\varphi \star \psi\right) + 1
    \end{align*}
    Por lo tanto
    \begin{align*}
        atom\left(\varphi\right) + atom\left(\psi\right) &\leq 2 \cdot con\left(\varphi \star \psi\right) + 1 \\
        atom\left(\varphi \star \psi\right) &\leq 2 \cdot con\left(\varphi \star \psi\right) + 1 \\
    \end{align*}
    Por lo que podemos concluir que
    \[
        atom\left(\varphi \star \psi\right) \leq 2 \cdot con\left(\varphi \star \psi\right) + 1
    \]
    Por lo tanto se cumple la propiedad.
\end{proof}
\subsubsection{Ejercicio}
\noindent
Demuestra mediante inducción estructural que se cumpla lo siguiente:
\[
    con\left(\varphi\right) < 2^{depth\left(\varphi\right)}
\]
\end{document}