\documentclass[a4paper]{article} 
\addtolength{\hoffset}{-2.25cm}
\addtolength{\textwidth}{4.5cm}
\addtolength{\voffset}{-3.25cm}
\addtolength{\textheight}{5cm}
\setlength{\parskip}{0pt}
\setlength{\parindent}{0in}

%----------------------------------------------------------------------------------------
%	PACKAGES AND OTHER DOCUMENT CONFIGURATIONS
%----------------------------------------------------------------------------------------

\usepackage{blindtext} % Package to generate dummy text
\usepackage{charter} % Use the Charter font
\usepackage[utf8]{inputenc} % Use UTF-8 encoding
\usepackage{microtype} % Slightly tweak font spacing for aesthetics
\usepackage[english, ngerman]{babel} % Language hyphenation and typographical rules
\usepackage{amsthm, amsmath, amssymb} % Mathematical typesetting
\usepackage{float} % Improved interface for floating objects
\usepackage[final, colorlinks = true, 
            linkcolor = black, 
            citecolor = black]{hyperref} % For hyperlinks in the PDF
\usepackage{graphicx, multicol} % Enhanced support for graphics
\usepackage{xcolor} % Driver-independent color extensions
\usepackage{marvosym, wasysym} % More symbols
\usepackage{rotating} % Rotation tools
\usepackage{censor} % Facilities for controlling restricted text
\usepackage{listings, style/lstlisting} % Environment for non-formatted code, !uses style file!
\usepackage{pseudocode} % Environment for specifying algorithms in a natural way
\usepackage{style/avm} % Environment for f-structures, !uses style file!
\usepackage{booktabs} % Enhances quality of tables
\usepackage{tikz-qtree} % Easy tree drawing tool
\tikzset{every tree node/.style={align=center,anchor=north},
         level distance=2cm} % Configuration for q-trees
\usepackage{style/btree} % Configuration for b-trees and b+-trees, !uses style file!
\usepackage[backend=biber,style=numeric,
            sorting=nyt]{biblatex} % Complete reimplementation of bibliographic facilities
\addbibresource{ecl.bib}
\usepackage{csquotes} % Context sensitive quotation facilities
\usepackage[yyyymmdd]{datetime} % Uses YEAR-MONTH-DAY format for dates
\renewcommand{\dateseparator}{-} % Sets dateseparator to '-'
\usepackage{fancyhdr} % Headers and footers
\pagestyle{fancy} % All pages have headers and footers
\fancyhead{}\renewcommand{\headrulewidth}{0pt} % Blank out the default header
\fancyfoot[L]{} % Custom footer text
\fancyfoot[C]{} % Custom footer text
\fancyfoot[R]{\thepage} % Custom footer text
\newcommand{\note}[1]{\marginpar{\scriptsize \textcolor{red}{#1}}} % Enables comments in red on margin

%----------------------------------------------------------------------------------------

\newcommand{\pow}[2]{#1^{#2}}
\newcommand{\supra}[1]{\textsuperscript{#1}}
\begin{document}

%-------------------------------
%	TITLE SECTION
%-------------------------------

\fancyhead[C]{}
\hrule \medskip % Upper rule
\begin{minipage}{0.295\textwidth} 
\raggedright
\footnotesize
Fausto David Hernández Jasso \hfill\\   
@FaustoJH \hfill\\
fausto.david.hernandez.jasso@ciencias.unam.mx
\end{minipage}
\begin{minipage}{0.4\textwidth} 
\centering 
\large 
Lógica Computacional I\\ 
\normalsize 
Tarea 01\\ 
\end{minipage}
\begin{minipage}{0.295\textwidth} 
\raggedleft
\today\hfill\\
\end{minipage}
\medskip\hrule 
\bigskip
\section{Ejercicio 1}
\noindent
Para las siguientes fórmulas
\begin{align*}
    & \left(r \lor p \lor q\right) \land \left(q \rightarrow p \rightarrow \neg q\right) \land \neg p
    & \left(\neg p \lor q \lor r \leftrightarrow \neg s \land p\right) \rightarrow q \lor \neg p
\end{align*}
\begin{itemize}
    \item Muestra el \textbf{árbol de sintaxis abstracta} para identificar el conectivo principal.
    \item Restaura todos los paréntesis.
\end{itemize}
\subsection{Árbol de Sintaxis Abstracta}
\subsubsection{\(\left(r \lor p \lor q\right) \land \left(q \rightarrow p \rightarrow \neg q\right) \land \neg p\)}
    \begin{tikzpicture}[
        tlabel/.style={pos=0.4,right=-1pt,font=\footnotesize\color{red!70!black}},
        level 1/.style={sibling distance=3cm},
        level 2/.style={sibling distance=1.2cm}, 
        level 3/.style={sibling distance=1cm}, 
        level distance=2cm,
    ]
    \node{\(\land\)}
    child {node {\(\lor\)}
        child {node {\(r\)}}
        child {node {\(\lor\)}
            child {node {\(p\)}}
            child {node {\(q\)}}}
    }
    child {node {\(\land\)}
        child {node {\(\rightarrow\)} 
            child {node {\(q\)}}
            child {node {\(\rightarrow\)}
                child {node {\(p\)}}
                child {node {\(\neg\)} 
                    child {node {\(r\)}}}}}
        child {node {\(\neg\)}
            child {node {\(p\)}}}};
    \end{tikzpicture}
\subsubsection{\(\left(\neg p \lor q \lor r \leftrightarrow \neg s \land p\right) \rightarrow q \lor \neg p\)}
    \begin{tikzpicture}[
        tlabel/.style={pos=0.4,right=-1pt,font=\footnotesize\color{red!70!black}},
        level 1/.style={sibling distance=3cm},
        level 2/.style={sibling distance=2.5cm}, 
        level 3/.style={sibling distance=1.8cm}, 
        level 4/.style={sibling distance=1cm}, 
        level distance=2cm,
    ]
        \node{\(\lor\)}
        child {
            node {\(\rightarrow\)}
            child {
                node {\(\leftrightarrow\)}
                child {
                    node {\(\lor\)}
                    child {
                        node {\(\neg\)}
                        child {
                            node {\(p\)}
                        }
                    }
                    child {
                        node {\(\lor\)}
                        child {
                            node {\(q\)}
                        }
                        child {
                            node {\(r\)}
                        }
                    }
                }
                child {
                    node {\(\lor\)}
                    child {
                        node {\(\neg\)}
                        child {
                            node {\(s\)}
                        }
                    }
                    child {
                        node {\(p\)}
                    }
                }
            }
            child {
                node {\(p\)}
            }}
        child {
            node {\(\neg\)}
            child {
                node {\(p\)}
            }
        };
    \end{tikzpicture}
    \subsubsection{Restaura todos los paréntesis}
    Veamos \(\left(r \lor p \lor q\right) \land \left(q \rightarrow p \rightarrow \neg q\right) \land \neg p\)
    \begin{align*}
        \left(r \lor p \lor q\right) &\land \left(q \rightarrow p \rightarrow \neg q\right) \land \neg p & \text{} \\
        \left(r \lor p \lor q\right) &\land \left(q \rightarrow p \rightarrow \left(\neg q\right)\right) \land \left(\neg p\right) & \text{Iniciamos con \(\neg\)} \\
        \left(r \lor \left(p \lor q\right)\right) &\land \left(q \rightarrow p \rightarrow \left(\neg q\right)\right) \land \left(\neg p\right) & \text{Iniciamos con \(\lor\)} \\
        \left(r \lor \left(p \lor q\right)\right) &\land \left(\left(q \rightarrow p \rightarrow \left(\neg q\right)\right) \land \left(\neg p\right)\right) & \text{Iniciamos con \(\land\)} \\
        \left(r \lor \left(p \lor q\right)\right) &\land \left(\left(q \rightarrow \left(p \rightarrow \left(\neg q\right)\right)\right) \land \left(\neg p\right)\right) & \text{Iniciamos con \(\rightarrow\)} \\
    \end{align*}
    Veamos \(\left(\neg p \lor q \lor r \leftrightarrow \neg s \land p\right) \rightarrow q \lor \neg p\)
    \begin{align*}
        \left(\neg p \lor q \lor r \leftrightarrow \neg s \land p\right) \rightarrow q \lor \neg p 
    \end{align*}
\end{document}