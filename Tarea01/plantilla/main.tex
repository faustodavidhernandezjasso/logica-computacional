\documentclass[a4paper]{article} 
\input{head}
\newcommand{\pow}[2]{#1^{#2}}
\newcommand{\supra}[1]{\textsuperscript{#1}}
\begin{document}

%-------------------------------
%	TITLE SECTION
%-------------------------------

\fancyhead[C]{}
\hrule \medskip % Upper rule
\begin{minipage}{0.295\textwidth} 
\raggedright
\footnotesize
Fausto David Hernández Jasso \hfill\\   
@FaustoJH \hfill\\
fausto.david.hernandez.jasso@ciencias.unam.mx
\end{minipage}
\begin{minipage}{0.4\textwidth} 
\centering 
\large 
Lógica Computacional I\\ 
\normalsize 
Tarea 01\\ 
\end{minipage}
\begin{minipage}{0.295\textwidth} 
\raggedleft
\today\hfill\\
\end{minipage}
\medskip\hrule 
\bigskip
\section{Ejercicio 1}
\noindent
Para las siguientes fórmulas
\begin{align*}
    & \left(r \lor p \lor q\right) \land \left(q \rightarrow p \rightarrow \neg q\right) \land \neg p
    & \left(\neg p \lor q \lor r \leftrightarrow \neg s \land p\right) \rightarrow q \lor \neg p
\end{align*}
\begin{itemize}
    \item Muestra el \textbf{árbol de sintaxis abstracta} para identificar el conectivo principal.
    \item Restaura todos los paréntesis.
\end{itemize}
\subsection{Árbol de Sintaxis Abstracta}
\subsubsection{\(\left(r \lor p \lor q\right) \land \left(q \rightarrow p \rightarrow \neg q\right) \land \neg p\)}
    \begin{tikzpicture}[
        tlabel/.style={pos=0.4,right=-1pt,font=\footnotesize\color{red!70!black}},
        level 1/.style={sibling distance=3cm},
        level 2/.style={sibling distance=1.2cm}, 
        level 3/.style={sibling distance=1cm}, 
        level distance=2cm,
    ]
    \node{\(\land\)}
    child {node {\(\lor\)}
        child {node {\(r\)}}
        child {node {\(\lor\)}
            child {node {\(p\)}}
            child {node {\(q\)}}}
    }
    child {node {\(\land\)}
        child {node {\(\rightarrow\)} 
            child {node {\(q\)}}
            child {node {\(\rightarrow\)}
                child {node {\(p\)}}
                child {node {\(\neg\)} 
                    child {node {\(r\)}}}}}
        child {node {\(\neg\)}
            child {node {\(p\)}}}};
    \end{tikzpicture}
\subsubsection{\(\left(\neg p \lor q \lor r \leftrightarrow \neg s \land p\right) \rightarrow q \lor \neg p\)}
    \begin{tikzpicture}[
        tlabel/.style={pos=0.4,right=-1pt,font=\footnotesize\color{red!70!black}},
        level 1/.style={sibling distance=3cm},
        level 2/.style={sibling distance=2.5cm}, 
        level 3/.style={sibling distance=1.8cm}, 
        level 4/.style={sibling distance=1cm}, 
        level distance=2cm,
    ]
        \node{\(\lor\)}
        child {
            node {\(\rightarrow\)}
            child {
                node {\(\leftrightarrow\)}
                child {
                    node {\(\lor\)}
                    child {
                        node {\(\neg\)}
                        child {
                            node {\(p\)}
                        }
                    }
                    child {
                        node {\(\lor\)}
                        child {
                            node {\(q\)}
                        }
                        child {
                            node {\(r\)}
                        }
                    }
                }
                child {
                    node {\(\lor\)}
                    child {
                        node {\(\neg\)}
                        child {
                            node {\(s\)}
                        }
                    }
                    child {
                        node {\(p\)}
                    }
                }
            }
            child {
                node {\(p\)}
            }}
        child {
            node {\(\neg\)}
            child {
                node {\(p\)}
            }
        };
    \end{tikzpicture}
    \subsubsection{Restaura todos los paréntesis}
    Veamos \(\left(r \lor p \lor q\right) \land \left(q \rightarrow p \rightarrow \neg q\right) \land \neg p\)
    \begin{align*}
        \left(r \lor p \lor q\right) &\land \left(q \rightarrow p \rightarrow \neg q\right) \land \neg p & \text{} \\
        \left(r \lor p \lor q\right) &\land \left(q \rightarrow p \rightarrow \left(\neg q\right)\right) \land \left(\neg p\right) & \text{Iniciamos con \(\neg\)} \\
        \left(r \lor \left(p \lor q\right)\right) &\land \left(q \rightarrow p \rightarrow \left(\neg q\right)\right) \land \left(\neg p\right) & \text{Iniciamos con \(\lor\)} \\
        \left(r \lor \left(p \lor q\right)\right) &\land \left(\left(q \rightarrow p \rightarrow \left(\neg q\right)\right) \land \left(\neg p\right)\right) & \text{Iniciamos con \(\land\)} \\
        \left(r \lor \left(p \lor q\right)\right) &\land \left(\left(q \rightarrow \left(p \rightarrow \left(\neg q\right)\right)\right) \land \left(\neg p\right)\right) & \text{Iniciamos con \(\rightarrow\)} \\
    \end{align*}
    Veamos \(\left(\neg p \lor q \lor r \leftrightarrow \neg s \land p\right) \rightarrow q \lor \neg p\)
    \begin{align*}
        \left(\neg p \lor q \lor r \leftrightarrow \neg s \land p\right) \rightarrow q \lor \neg p 
    \end{align*}
\end{document}