\documentclass[a4paper]{article} 
\input{head}
\newcommand{\pow}[2]{#1^{#2}}
\newcommand{\supra}[1]{\textsuperscript{#1}}
\begin{document}

%-------------------------------
%	TITLE SECTION
%-------------------------------

\fancyhead[C]{}
\hrule \medskip % Upper rule
\begin{minipage}{0.35\textwidth} 
\raggedright
\footnotesize
Fausto David Hernández Jasso \hfill\\   
\texttt{@FaustoJH} \hfill\\
\href{mailto:fausto.david.hernandez.jasso@ciencias.unam.mx}{fausto.david.hernandez.jasso@ciencias.unam.mx}
\end{minipage}
\begin{minipage}{0.4\textwidth} 
\centering 
\large 
Lógica Ecuacional\\ 
\normalsize 
Ejemplos\\ 
\end{minipage}
\begin{minipage}{0.24\textwidth} 
\raggedleft
\today\hfill\\
\end{minipage}
\medskip\hrule
\bigskip
\section{Anillos}
\noindent
Una anillo es una estructura algebraica \(\mathfrak{R} = (R, +, \cdot, 0_{R})\),
donde \(+\) y \(\cdot\) son operaciones binarias que cumplen lo siguiente:
\begin{itemize}
    \item \(+\) es conmutativa,
    \item \(+\) es asociativa,
    \item \(\cdot\) es asociativa,
    \item Para todo elemento \(x \in R\) existe un elemento \(-x \in R\) tal que 
    \(x + (-x) = 0_{R}\),
    \item Para todo elemento \(x \in R\) se cumple que \(x + 0_{R} = x\),
    \item \(\cdot\) se distribuye sobre \(+\)
\end{itemize}
Lo anterior lo podemos representar con el siguiente conjunto de ecuaciones, al que
nombraremos \(\mathcal{A}\).
\newline 
Sean \(x, y, z\) cualesquiera elementos de \(R\) se tiene que:
\begin{align*}
    x + 0_{R} &= x & &A1 \\
    x + y &= y + z & &A2 \\
    x + (-x) &= 0_{R} & &A3 \\
    (x + y) + z &= x + (y + z) & &A4 \\
    (x \cdot y) \cdot z &= x \cdot (y \cdot z) & &A5 \\
    x \cdot (y + z) &= x \cdot y + x \cdot z & &A6 \\
    (x + y) \cdot z &= x \cdot z + y \cdot z & &A7
\end{align*}
Cuando no haya ambigüedad denotaremos a \(0_{R}\) como simplemente \(0\).
\newline
Al conjunto de ecuaciones \(\mathcal{A}\) lo llamaremos de ahora en adelante
``base de datos''.
\newline
Algunos ejemplos de anillos son:
\begin{itemize}
    \item \((\mathbb{Z}, +_{\mathbb{Z}}, \cdot_{\mathbb{Z}}, 0_{\mathbb{Z}})\).
    \item Los polinomios con coeficientes en los números complejos con la suma,
    la multiplicación de polinomios usuales y el elemento neutro de la suma de
    polinomios es el polinomio \(0\).
    \item Las matrices de \(2 \times 2\) con coeficientes reales con la suma y 
    la multiplicación de matrices usuales y el elemento neutro de la suma
    es la matriz cuyas entradas son \(0\).
\end{itemize}
Un teorema muy simple en teoría de anillos es el siguiente:
\newline
Sea un anillo \(\mathfrak{R} = (R, +, \cdot, 0)\) y sea \(a \in R\) entonces se cumple 
la siguiente igualdad:
\newline 
\[
    a \cdot 0 = 0
\]
\subsection{Demostración usual}
\label{dem_usual}
\begin{proof}
    \begin{align*}
        a \cdot 0 &= a \cdot (0 + 0) & &\text{por } A1 \\
                  &= a \cdot 0 + a \cdot 0 & &\text{por } A6 \\
    \end{align*}
    Por transitividad de la igualdad tenemos que
    \begin{align*}
        a \cdot 0 &= a \cdot 0 + a \cdot 0 & & \\
        a \cdot 0 + (-a \cdot 0) &= (a \cdot 0 + a \cdot 0) + (-a \cdot 0) & &\text{sumando \(-a \cdot 0\)} \\
        0 &= (a \cdot 0 + a \cdot 0) + (-a \cdot 0) & &\text{por } A3 \\
        0 &= a \cdot 0 + (a \cdot 0 + (-a \cdot 0)) & &\text{por } A4 \\
        0 &= a \cdot 0 + 0 & &\text{por } A3 \\
        0 &= a \cdot 0 & &\text{por } A1
    \end{align*}
    Por simetría de la igualdad tenemos que
    \begin{align*}
        a \cdot 0 &= 0 & &
    \end{align*}
\end{proof}
\subsection{Reglas de inferencia de la lógica ecuacional}
\noindent
Antes de entrar a la demostración con lógica ecuacional recordemos 
las reglas de inferencia:
\begin{center}
    \begin{prooftree}
        \hypo{}
        \infer1[Hyp]{H}
    \end{prooftree}
    \qquad
    \begin{prooftree}
        \hypo{}
        \infer1[Refl]{t = t}
    \end{prooftree}
    \qquad
    \begin{prooftree}
        \hypo{s = t}
        \infer1[Sym]{t = s}
    \end{prooftree}
    \qquad
    \begin{prooftree}
        \hypo{t = r}
        \hypo{r = s}
        \infer2[Trans]{t = s}
    \end{prooftree}
    \qquad
    \begin{prooftree}
        \hypo{t = s}
        \infer1[Inst]{t\sigma = s\sigma}
    \end{prooftree}
    \qquad
    \begin{prooftree}
        \hypo{t_{1} = s_{1}}
        \hypo{\dotsc}
        \hypo{t_{n} = s_{n}}
        \infer3[Congr]{f(t_{1}, \dotsc, t_{n}) = f(s_{1}, \dotsc, s_{n})}
    \end{prooftree}
\end{center}
\noindent
Al conjunto anterior de reglas de inferencia es un ejemplo de un sistema de deducción.
\subsection{Ejemplos de instancias de las reglas}
\noindent
Recordemos la definición de derivación dada en la \href{https://drive.google.com/drive/folders/1L7Ha_HJKRPGi3keZSfXxgqD4pyj1btBa}{nota 9} del curso.
\newline 
Una deducción o derivación de un enunciado \(\mathcal{J}\) usando un sistema de deducción,
es una secuencia finita \(\Pi = \langle \mathcal{J}_{1}, \dotsc,\mathcal{J}_{k} \rangle\)
tal que \(\mathcal{J}_{k} = \mathcal{J}\) y cada \(\mathcal{J}_{i}\), con \(i \in \{1, \dotsc, k\}\)
es una instancia de una regla del sistema de deducción.
\newline 
Daremos ejemplos de instancias de cada una de las reglas de inferencia. Tomemos 
como base la demostración usual del teorema en la sección \ref{dem_usual} y la 
``base de datos''.
\subsubsection{Ejemplo de la regla Hyp}
\noindent
\begin{prooftree}
    \hypo{}
        \infer1[Hyp]{x + y = y + x}
\end{prooftree}
\newline
La instancias de la regla Hyp siempre serán alguna de las ecuaciones de nuestra 
``base de datos''. En este ejemplo en concreto las instancias de la regla Hyp
solamente pueden ser los axiomas de anillos definidos al inicio de este documento.
\subsubsection{Ejemplo de la regla Refl}
\noindent
\begin{prooftree}
    \hypo{}
        \infer1[Refl]{- 0\cdot a = - 0\cdot a}
\end{prooftree}
\subsubsection{Ejemplo de la regla Sym}
\noindent
Tomando como ejemplo la demostración usual del teorema en la sección 
\ref{dem_usual} una instancia de la regla Sym es la siguiente:
\newline 
\begin{prooftree}
    \hypo{0 = a \cdot 0}
    \infer1[Sym]{a \cdot 0 = 0}
\end{prooftree}
\subsubsection{Ejemplo de la regla Trans}
\noindent
Tomando como ejemplo la demostración usual del teorema en la sección 
\ref{dem_usual} una instancia de la regla Trans es la siguiente:
\newline 
\begin{prooftree}
    \hypo{a \cdot 0 = a \cdot (0 + 0)}
    \hypo{a \cdot (0 + 0) = a \cdot 0 + a \cdot 0}
    \infer2[Trans]{a \cdot 0 = a \cdot 0 + a \cdot 0}
\end{prooftree}
\subsubsection{Ejemplo de la regla Inst}
\noindent
\begin{prooftree}
    \hypo{x + 0 = x}
    \infer1[Inst]{0 + 0 = 0}
\end{prooftree}
\newline
Es decir aquí realizamos una sustitución cambiamos a \(x\) por \(0\).

\subsubsection{Ejemplo de la regla Congr}
\noindent
Tomando como ejemplo la demostración usual del teorema en la sección 
\ref{dem_usual} una instancia de la regla Congr es la siguiente:
\newline 
\begin{prooftree}
    \hypo{a \cdot 0 = a \cdot 0 + a \cdot 0}
    \hypo{- 0\cdot a = - 0\cdot a}
    \infer2[Congr]{(a \cdot 0) + (- 0\cdot a) = (a \cdot 0 + a \cdot 0) + (- 0\cdot a)}
\end{prooftree}
\newline 
En éste ejemplo en concreto podemos aplicar las regla Congr con dos operaciones 
\textit{(funciones)}, las cuales son: \(+\) y \(\cdot\).

\subsection{Demostración con lógica ecuacional}
\noindent
Ahora que hemos dado un ejemplo de instancias de cada una las reglas comencemos
a realizar la demostración con lógica ecuacional de la igualdad \(a \cdot 0 = 0\).
\newline 
En la primera igualdad de la \ref{dem_usual} utilizamos implícitamente cinco reglas
la regla de Refl, Hyp, Inst, Sym y Congr.
\newline 
Para lograr la igualdad \(a \cdot 0 = a \cdot (0 + 0)\) es necesario lo siguiente en lógica
ecuacional:
\begin{align*}
    &\text{1.} & a &= a & &\text{Refl} \\
    &\text{2.} & x + 0 &= x & &\text{Hyp }A1 \\
    &\text{3.} & 0 + 0 &= 0 & &\text{Inst (2)} \\
    &\text{4.} & 0 &= 0 + 0 & &\text{Sym (3)} \\
    &\text{5.} & a \cdot 0 &= a \cdot (0 + 0) & &\text{Congr \(\cdot\) (1)(4)}
\end{align*}
Para obtener la ecuación \(a \cdot (0 + 0) = a \cdot 0 + a \cdot 0\) utilizamos 
implícitamente las reglas de Hyp e Inst:
\begin{align*}
    &\text{6.} & x \cdot (y + z) &= x \cdot y + x \cdot z & &\text{Hyp } A6 \\
    &\text{7.} & a \cdot (0 + 0) &= a \cdot 0 + a \cdot 0 & &\text{Inst (6)}
\end{align*}
Para obtener la ecuación \(a \cdot 0 = a \cdot 0 + a \cdot 0\) utilizamos 
implícitamente la regla Trans:
\begin{align*}
    &\text{8.} & a \cdot 0 &= a \cdot 0 + a \cdot 0 & &\text{Trans (5)(7)}
\end{align*}
Para obtener la ecuación \((a \cdot 0) + (-a \cdot 0)  = (a \cdot 0 + a \cdot 0) + (-a \cdot 0)\)
utilizamos implícitamente las reglas de Refl y Congr:
\begin{align*}
    &\text{9.} & -a \cdot 0 &= -a \cdot 0 & &\text{Refl} \\
    &\text{10.} & (a \cdot 0) + (-a \cdot 0) &= (a \cdot 0 + a \cdot 0) + (-a \cdot 0) & &\text{Congr + (8)(9)}
\end{align*}
Para obtener la ecuación \((a \cdot 0) + (-a \cdot 0) = 0\) utilizamos implícitamente las reglas de 
Hyp e Inst:
\begin{align*}
    &\text{11.} & x + (-x) &= 0 & &\text{Hyp }A3 \\
    &\text{12.} & (a \cdot 0) + (-a \cdot 0) &= 0 & &\text{Inst (11)}
\end{align*}
Para obtener la ecuación \(0 = (a \cdot 0 + a \cdot 0) + (-a \cdot 0)\)
utilizamos implícitamente las reglas de Sym y Trans:
\begin{align*}
    &\text{13.} & 0 &= (a \cdot 0) + (-a \cdot 0) & &\text{Sym (12)} \\
    &\text{14.} & 0 &= (a \cdot 0 + a \cdot 0) + (-a \cdot 0)  & &\text{Trans (13)(10)}
\end{align*}
Para obtener la ecuación \(0 = a \cdot 0 + (a \cdot 0 + (-a \cdot 0))\)
utilizamos implícitamente las reglas de Hyp, Inst y Trans:
\begin{align*}
    &\text{15.} & (x + y) + z &= x + (y + z) & &\text{Hyp } A2 \\
    &\text{16.} &  (a \cdot 0 + a \cdot 0) + (-a \cdot 0) &= a \cdot 0 + (a \cdot 0 + (-a \cdot 0))  & &\text{Inst (15)} \\
    &\text{17.} &  0 &= a \cdot 0 + (a \cdot 0 + (-a \cdot 0))  & &\text{Trans (14)(16)}
\end{align*}
Para obtener la ecuación \(a \cdot 0 + (a \cdot 0 + (-a \cdot 0)) = a \cdot 0 + 0\)
utilizamos implícitamente las reglas de Refl y Congr
\begin{align*}
    &\text{18.} & a \cdot 0 &= a \cdot 0 & &\text{Refl} \\
    &\text{19.} & a \cdot 0 + (a \cdot 0 + (-a \cdot 0)) &= a \cdot 0 + 0  & &\text{Congr + (18)(12)}
\end{align*}
Para obtener la ecuación \(0 = a \cdot 0 + 0\) utilizamos implícitamente la regla de Trans
\begin{align*}
    &\text{20.} & 0 &= a \cdot 0 + 0 & &\text{Trans (17)(19)}
\end{align*}
Para obtener la ecuación \(a \cdot 0 = 0\) utilizamos implícitamente las reglas de 
Sym, Inst y Trans:
\begin{align*}
    &\text{21.} & a \cdot 0 + 0 &= 0 & &\text{Sym (20)} \\
    &\text{22.} & a \cdot 0 + 0 &= a \cdot 0 & &\text{Inst (2)} \\
    &\text{23.} & a \cdot 0 &= a \cdot 0 + 0 & &\text{Sym (22)} \\
    &\text{24.} & a \cdot 0 &= 0 & &\text{Trans (23)(21)}
\end{align*}
Por lo tanto queda demostrada la igualdad.
\newline 
La demostración sin el análisis anterior quedaría de la siguiente manera:
\begin{align*}
    &\text{1.} & a &= a & &\text{Refl} \\
    &\text{2.} & x + 0 &= x & &\text{Hyp} \\
    &\text{3.} & 0 + 0 &= 0 & &\text{Inst (2)} \\
    &\text{4.} & 0 &= 0 + 0 & &\text{Sym (3)} \\
    &\text{5.} & a \cdot 0 &= a \cdot (0 + 0) & &\text{Congr \(\cdot\) (1)(4)} \\
    &\text{6.} & x \cdot (y + z) &= x \cdot y + x \cdot z & &\text{Hyp } A6 \\
    &\text{7.} & a \cdot (0 + 0) &= a \cdot 0 + a \cdot 0 & &\text{Inst (6)} \\
    &\text{8.} & a \cdot 0 &= a \cdot 0 + a \cdot 0 & &\text{Trans (5)(7)} \\
    &\text{9.} & -a \cdot 0 &= -a \cdot 0 & &\text{Refl} \\
    &\text{10.} & (a \cdot 0) + (-a \cdot 0) &= (a \cdot 0 + a \cdot 0) + (-a \cdot 0) & &\text{Congr + (8)(9)} \\
    &\text{11.} & x + (-x) &= 0 & &\text{Hyp } A3\\
    &\text{12.} & (a \cdot 0) + (-a \cdot 0) &= 0 & &\text{Inst (11)} \\
    &\text{13.} & 0 &= (a \cdot 0) + (-a \cdot 0) & &\text{Sym (12)} \\
    &\text{14.} & 0 &= (a \cdot 0 + a \cdot 0) + (-a \cdot 0)  & &\text{Trans (13)(10)} \\
    &\text{15.} & (x + y) + z &= x + (y + z) & &\text{Hyp } A2 \\
    &\text{16.} &  (a \cdot 0 + a \cdot 0) + (-a \cdot 0) &= a \cdot 0 + (a \cdot 0 + (-a \cdot 0))  & &\text{Inst (15)} \\
    &\text{17.} &  0 &= a \cdot 0 + (a \cdot 0 + (-a \cdot 0))  & &\text{Trans (14)(16)} \\
    &\text{18.} & a \cdot 0 &= a \cdot 0 & &\text{Refl} \\
    &\text{19.} & a \cdot 0 + (a \cdot 0 + (-a \cdot 0)) &= a \cdot 0 + 0  & &\text{Congr + (18)(12)} \\
    &\text{20.} & 0 &= a \cdot 0 + 0 & &\text{Trans (17)(19)} \\
    &\text{21.} & a \cdot 0 + 0 &= 0 & &\text{Sym (20)} \\
    &\text{22.} & a \cdot 0 + 0 &= a \cdot 0 & &\text{Inst (2)} \\
    &\text{23.} & a \cdot 0 &= a \cdot 0 + 0 & &\text{Sym (22)} \\
    &\text{24.} & a \cdot 0 &= 0 & &\text{Trans (23)(21)}
\end{align*}
Sin el análisis anterior la demostración parecería sacada de la manga o al menos 
muy ``esotérica''.
\newline 
Durante el análisis mencionamos la palabra implícitamente cada vez que derivamos 
una ecuación de la demostración usual en la sección \ref{dem_usual} porque son 
reglas que estamos usando también en la demostración usual pero por convención 
no las escribimos porque se dan por hecho que el lector sabe esos detalles.
\newline 
La prueba anterior en su forma de árbol de derivación se construye de abajo hacia arriba,
partimos de que la raíz es \(a\cdot0 = 0\) y a partir de ahí vamos aplicando las reglas 
de inferencia de abajo hacia arriba. En el caso de las reglas Trans y Congr que en la
demostración en forma de lista vienen de dos ecuaciones anteriores, la primera corresponde 
a la rama izquierda del árbol, mientras que la segunda corresponde a su rama derecha.
Ejemplificando:
\newline 
\begin{prooftree}
    \hypo{}
    \ellipsis{}{a \cdot 0 + 0 = a \cdot 0}
    \infer1[Sym]{a \cdot 0 = a \cdot 0 + 0}
    \hypo{}
    \ellipsis{}{0 = a \cdot 0 + 0}
    \infer1[Sym]{a \cdot 0 + 0 = 0}
    \infer2[Trans]{a \cdot 0 = 0}
\end{prooftree}
\newline 
Sucede lo análogo cuando venimos de una regla Congr.
\subsection{Conclusión}
\noindent
Tanto la demostración usual como la demostración con lógica ecuacional son equivalentes y 
de hecho hicimos explícito el hecho de que al momento de demostrar de la forma usual 
es decir haciendo la secuencia de axiomas implícitamente estamos usando las reglas de la 
lógica ecuacional.
\subsection{Ejercicios de práctica}
\noindent
Para practicar las demostraciones en lógica ecuacional puedes realizar las siguientes 
demostraciones en lógica ecuacional:
\begin{itemize}
    \item \(-a = (-1) \cdot a\)
    \item \((-a)(-b) = ab\)
\end{itemize}
Se recomienda ampliamente hacer la demostración de sucesión de axiomas y de ahí
guiarse para hacer la demostración en lógica ecuacional.
\end{document}