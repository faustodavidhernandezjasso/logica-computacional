\documentclass[a4paper]{article} 
\input{head}
\newcommand{\pow}[2]{#1^{#2}}
\newcommand{\supra}[1]{\textsuperscript{#1}}
\begin{document}

%-------------------------------
%	TITLE SECTION
%-------------------------------

\fancyhead[C]{}
\hrule \medskip % Upper rule
\begin{minipage}{0.295\textwidth} 
\raggedright
\footnotesize
Fausto David Hernández Jasso \hfill\\   
@FaustoJH \hfill\\
fausto.david.hernandez.jasso@ciencias.unam.mx
\end{minipage}
\begin{minipage}{0.4\textwidth} 
\centering 
\large 
Lógica Computacional I\\ 
\normalsize 
Tarea 03\\ 
\end{minipage}
\begin{minipage}{0.295\textwidth} 
\raggedleft
\today\hfill\\
\end{minipage}
\medskip\hrule 
\bigskip
\section{Ejercicio 1}
Demuestra usando el algoritmo DPLL con búsqueda hacia atrás si los siguientes conjuntos son satisfacibles:
\begin{enumerate}
    \item \(\{p \rightarrow s \lor t, m \land r \leftrightarrow s, s, m\}\)
    \item \(\{k \leftrightarrow r \lor p, k \land p \rightarrow \neg \left(s \land p\right), \neg \left(\neg k \lor r\right), \neg s \}\)
\end{enumerate}
\end{document}