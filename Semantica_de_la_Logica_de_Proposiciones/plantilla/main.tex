\documentclass[a4paper]{article} 
\input{head}
\newcommand{\pow}[2]{#1^{#2}}
\newcommand{\supra}[1]{\textsuperscript{#1}}
\begin{document}

%-------------------------------
%	TITLE SECTION
%-------------------------------

\fancyhead[C]{}
\hrule \medskip % Upper rule
\begin{minipage}{0.35\textwidth} 
\raggedright
\footnotesize
Fausto David Hernández Jasso \hfill\\   
@FaustoJH \hfill\\
fausto.david.hernandez.jasso@ciencias.unam.mx
\end{minipage}
\begin{minipage}{0.4\textwidth} 
\centering 
\large 
Lógica Computacional I\\ 
\normalsize 
Semántica de la Lógica de Proposiciones\\ 
\end{minipage}
\begin{minipage}{0.24\textwidth} 
\raggedleft
\today\hfill\\
\end{minipage}
\medskip\hrule 
\bigskip
\section{Significado de los conectivos lógicos}
\subsection{Negación}
\begin{itemize}
    \item Símbolo utilizado:  $\lnot$ 
    \item Correspondencia con el español: \textbf{No}, no es cierto que, es falso que, 
     etc.
    \item Otros símbolos: $\sim \varphi, \ \overline{\varphi}$. 
\end{itemize}
\subsubsection{Tabla de verdad}
\begin{center}
    \begin{tabular}{cc}
    $\varphi$ & $\neg \varphi$\\
    \midrule
    1 & 0 \\
    0 & 1
    \end{tabular}
\end{center}
\subsection{Disyunción}
\subsubsection{Descripción}
\noindent
La \textbf{disyunción} de las fórmulas $\varphi,\psi$ es la fórmula
$\varphi \lor \psi$. 
\newline
Las fórmulas $\varphi,\psi$ se llaman \textbf{disyuntos}.
\begin{itemize}
    \item Símbolo utilizado:  $\lor$ 
    \item Correspondencia con el español: ó.
    \item Otros símbolos: \(\varphi + \psi, \varphi \ | \ \psi\).
\end{itemize}
\subsubsection{Tabla de verdad}
\begin{center}
    \begin{tabular}{ccc}
    $\varphi$ & \(\psi\) & $\varphi \lor \psi$\\
    \midrule
    1 & 1 & 1 \\
    1 & 0 & 1 \\
    0 & 1 & 1 \\
    0 & 0 & 0 \\
    \end{tabular}
\end{center}
\subsection{Conjunción}
\subsubsection{Descripción}
La \textbf{conjunción} de las fórmulas $\varphi,\psi$ es la fórmula
$\varphi \land \psi$. Las fórmulas $\varphi ,\psi$ se llaman 
\textbf{conyuntos}.
\begin{itemize}
    \item Símbolo utilizado: \(\land\)
    \item Correspondencia con el español: y, pero
    \item Otros símbolos: \(\varphi \land \psi\), \(\varphi \cdot \psi\) ó \(\varphi \psi\)
\end{itemize}
\subsubsection{Tabla de verdad}
\begin{center}
    \begin{tabular}{ccc}
    $\varphi$ & \(\psi\) & $\varphi \land \psi$\\
    \midrule
    1 & 1 & 1 \\
    1 & 0 & 0 \\
    0 & 1 & 0 \\
    0 & 0 & 0 \\
    \end{tabular}
\end{center}
\subsection{Implicación}
\subsubsection{Descripción}
La \textbf{implicación} o \textbf{condicional} de las fórmulas $\varphi ,\psi$ es la 
fórmula $\varphi \rightarrow \psi$. 
La fórmula $\varphi $ es el \emph{antecedente}\index{antecedente} y la
fórmula $\psi$ es el \emph{consecuente} de la implicación. 
\begin{itemize}
    \item Símbolo utilizado: \(\rightarrow\)
    \item Correspondencia con el español: $\varphi \rightarrow \psi$ significa: si $\varphi $
    entonces $\psi$; $\psi$, si $\varphi $; $\varphi $ sólo si $\psi$; $\varphi $ es 
    condición suficiente para $\psi$; $\psi$ es condición necesaria para 
    $\varphi $.
    \item Otros símbolos: \(\varphi \Rightarrow \psi\), \(\varphi \supset \psi\)
\end{itemize}
\subsubsection{Tabla de verdad}
\begin{center}
    \begin{tabular}{ccc}
    $\varphi$ & \(\psi\) & $\varphi \rightarrow  \psi$\\
    \midrule
    1 & 1 & 1 \\
    1 & 0 & 0 \\
    0 & 1 & 1 \\
    0 & 0 & 1 \\
    \end{tabular}
\end{center}
\subsection{Doble implicación}
\subsubsection{Descripción}
\noindent
La equivalencia o bicondicional de las fórmulas $\varphi ,\psi$ es la fórmula
$\varphi \leftrightarrow \psi$. 
\begin{itemize}
    \item Símbolo utilizado:  \(\leftrightarrow\)
    \item Correspondencia con el español: $\varphi$ es equivalente a $\psi$; 
    $\varphi$ si y sólo si $\psi$; $\varphi$ es condición necesaria y suficiente 
    para $\psi$.
    \item Otros símbolos: \(\varphi \Leftrightarrow \psi\), \(\varphi \equiv \psi\)
\end{itemize}
\subsubsection{Tabla de verdad}
\begin{center}
    \begin{tabular}{ccc}
    $\varphi$ & \(\psi\) & $\varphi \leftrightarrow \psi$\\
    \midrule
    1 & 1 & 1 \\
    1 & 0 & 0 \\
    0 & 1 & 0 \\
    0 & 0 & 1 \\
    \end{tabular}
\end{center}
\end{document}