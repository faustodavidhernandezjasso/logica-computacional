\documentclass[a4paper]{article} 
\addtolength{\hoffset}{-2.25cm}
\addtolength{\textwidth}{4.5cm}
\addtolength{\voffset}{-3.25cm}
\addtolength{\textheight}{5cm}
\setlength{\parskip}{0pt}
\setlength{\parindent}{0in}

%----------------------------------------------------------------------------------------
%	PACKAGES AND OTHER DOCUMENT CONFIGURATIONS
%----------------------------------------------------------------------------------------

\usepackage{blindtext} % Package to generate dummy text
\usepackage{charter} % Use the Charter font
\usepackage[utf8]{inputenc} % Use UTF-8 encoding
\usepackage{microtype} % Slightly tweak font spacing for aesthetics
\usepackage[english, ngerman]{babel} % Language hyphenation and typographical rules
\usepackage{amsthm, amsmath, amssymb} % Mathematical typesetting
\usepackage{float} % Improved interface for floating objects
\usepackage[final, colorlinks = true, 
            linkcolor = black, 
            citecolor = black]{hyperref} % For hyperlinks in the PDF
\usepackage{graphicx, multicol} % Enhanced support for graphics
\usepackage{xcolor} % Driver-independent color extensions
\usepackage{marvosym, wasysym} % More symbols
\usepackage{rotating} % Rotation tools
\usepackage{censor} % Facilities for controlling restricted text
\usepackage{listings, style/lstlisting} % Environment for non-formatted code, !uses style file!
\usepackage{pseudocode} % Environment for specifying algorithms in a natural way
\usepackage{style/avm} % Environment for f-structures, !uses style file!
\usepackage{booktabs} % Enhances quality of tables
\usepackage{tikz-qtree} % Easy tree drawing tool
\tikzset{every tree node/.style={align=center,anchor=north},
         level distance=2cm} % Configuration for q-trees
\usepackage{style/btree} % Configuration for b-trees and b+-trees, !uses style file!
\usepackage[backend=biber,style=numeric,
            sorting=nyt]{biblatex} % Complete reimplementation of bibliographic facilities
\addbibresource{ecl.bib}
\usepackage{csquotes} % Context sensitive quotation facilities
\usepackage[yyyymmdd]{datetime} % Uses YEAR-MONTH-DAY format for dates
\renewcommand{\dateseparator}{-} % Sets dateseparator to '-'
\usepackage{fancyhdr} % Headers and footers
\pagestyle{fancy} % All pages have headers and footers
\fancyhead{}\renewcommand{\headrulewidth}{0pt} % Blank out the default header
\fancyfoot[L]{} % Custom footer text
\fancyfoot[C]{} % Custom footer text
\fancyfoot[R]{\thepage} % Custom footer text
\newcommand{\note}[1]{\marginpar{\scriptsize \textcolor{red}{#1}}} % Enables comments in red on margin

%----------------------------------------------------------------------------------------

\newcommand{\pow}[2]{#1^{#2}}
\newcommand{\supra}[1]{\textsuperscript{#1}}
\begin{document}

%-------------------------------
%	TITLE SECTION
%-------------------------------

\fancyhead[C]{}
\hrule \medskip % Upper rule
\begin{minipage}{0.35\textwidth} 
\raggedright
\footnotesize
Fausto David Hernández Jasso \hfill\\   
@FaustoJH \hfill\\
fausto.david.hernandez.jasso@ciencias.unam.mx
\end{minipage}
\begin{minipage}{0.4\textwidth} 
\centering 
\large 
Lógica Computacional I\\ 
\normalsize 
Semántica de la Lógica de Proposiciones\\ 
\end{minipage}
\begin{minipage}{0.24\textwidth} 
\raggedleft
\today\hfill\\
\end{minipage}
\medskip\hrule 
\bigskip
\section{Significado de los conectivos lógicos}
\subsection{Negación}
\begin{itemize}
    \item Símbolo utilizado:  $\lnot$ 
    \item Correspondencia con el español: \textbf{No}, no es cierto que, es falso que, 
     etc.
    \item Otros símbolos: $\sim \varphi, \ \overline{\varphi}$. 
\end{itemize}
\subsubsection{Tabla de verdad}
\begin{center}
    \begin{tabular}{cc}
    $\varphi$ & $\neg \varphi$\\
    \midrule
    1 & 0 \\
    0 & 1
    \end{tabular}
\end{center}
\subsection{Disyunción}
\subsubsection{Descripción}
\noindent
La \textbf{disyunción} de las fórmulas $\varphi,\psi$ es la fórmula
$\varphi \lor \psi$. 
\newline
Las fórmulas $\varphi,\psi$ se llaman \textbf{disyuntos}.
\begin{itemize}
    \item Símbolo utilizado:  $\lor$ 
    \item Correspondencia con el español: ó.
    \item Otros símbolos: \(\varphi + \psi, \varphi \ | \ \psi\).
\end{itemize}
\subsubsection{Tabla de verdad}
\begin{center}
    \begin{tabular}{ccc}
    $\varphi$ & \(\psi\) & $\varphi \lor \psi$\\
    \midrule
    1 & 1 & 1 \\
    1 & 0 & 1 \\
    0 & 1 & 1 \\
    0 & 0 & 0 \\
    \end{tabular}
\end{center}
\subsection{Conjunción}
\subsubsection{Descripción}
La \textbf{conjunción} de las fórmulas $\varphi,\psi$ es la fórmula
$\varphi \land \psi$. Las fórmulas $\varphi ,\psi$ se llaman 
\textbf{conyuntos}.
\begin{itemize}
    \item Símbolo utilizado: \(\land\)
    \item Correspondencia con el español: y, pero
    \item Otros símbolos: \(\varphi \land \psi\), \(\varphi \cdot \psi\) ó \(\varphi \psi\)
\end{itemize}
\subsubsection{Tabla de verdad}
\begin{center}
    \begin{tabular}{ccc}
    $\varphi$ & \(\psi\) & $\varphi \land \psi$\\
    \midrule
    1 & 1 & 1 \\
    1 & 0 & 0 \\
    0 & 1 & 0 \\
    0 & 0 & 0 \\
    \end{tabular}
\end{center}
\subsection{Implicación}
\subsubsection{Descripción}
La \textbf{implicación} o \textbf{condicional} de las fórmulas $\varphi ,\psi$ es la 
fórmula $\varphi \rightarrow \psi$. 
La fórmula $\varphi $ es el \emph{antecedente}\index{antecedente} y la
fórmula $\psi$ es el \emph{consecuente} de la implicación. 
\begin{itemize}
    \item Símbolo utilizado: \(\rightarrow\)
    \item Correspondencia con el español: $\varphi \rightarrow \psi$ significa: si $\varphi $
    entonces $\psi$; $\psi$, si $\varphi $; $\varphi $ sólo si $\psi$; $\varphi $ es 
    condición suficiente para $\psi$; $\psi$ es condición necesaria para 
    $\varphi $.
    \item Otros símbolos: \(\varphi \Rightarrow \psi\), \(\varphi \supset \psi\)
\end{itemize}
\subsubsection{Tabla de verdad}
\begin{center}
    \begin{tabular}{ccc}
    $\varphi$ & \(\psi\) & $\varphi \rightarrow  \psi$\\
    \midrule
    1 & 1 & 1 \\
    1 & 0 & 0 \\
    0 & 1 & 1 \\
    0 & 0 & 1 \\
    \end{tabular}
\end{center}
\subsection{Doble implicación}
\subsubsection{Descripción}
\noindent
La equivalencia o bicondicional de las fórmulas $\varphi ,\psi$ es la fórmula
$\varphi \leftrightarrow \psi$. 
\begin{itemize}
    \item Símbolo utilizado:  \(\leftrightarrow\)
    \item Correspondencia con el español: $\varphi$ es equivalente a $\psi$; 
    $\varphi$ si y sólo si $\psi$; $\varphi$ es condición necesaria y suficiente 
    para $\psi$.
    \item Otros símbolos: \(\varphi \Leftrightarrow \psi\), \(\varphi \equiv \psi\)
\end{itemize}
\subsubsection{Tabla de verdad}
\begin{center}
    \begin{tabular}{ccc}
    $\varphi$ & \(\psi\) & $\varphi \leftrightarrow \psi$\\
    \midrule
    1 & 1 & 1 \\
    1 & 0 & 0 \\
    0 & 1 & 0 \\
    0 & 0 & 1 \\
    \end{tabular}
\end{center}
\section{Semántica formal de los conectivos}
\subsection{Semántica formal de los conectivos lógicos}
\subsubsection{Tipo \texttt{Bool}}
El tipo de valores booleanos denotado como \text{\texttt{Bool}} se define como 
\text{\texttt{Bool}}$=\{0,1\}$.
\subsubsection{Estado}
Un estado o asignación de las variables (proposicionales) es
una función
\[
    \mathcal{I} \ : \ VarP \to \text{\texttt{Bool}}
\]
Dadas $n$ variables proposicionales existen $2^n$ estados distintos
para estas variables. Lo anterior tiene justificación a través del 
\textbf{principio de la multiplicación}. Supongamos que tenemos 
una variable entonces por el \textbf{principio de tercero excluido}, 
sabemos que sólo puede tener dos valores \(\top\) o \(\bot\). Sí tenemos 
dos variables cada una a su vez tendrá dos posibles valores los cuales serán
\(\top\) o \(\bot\), y tendremos cuatro posibles estados cuando ambas sean 
\(\top\), cuando ambas sean \(\bot\) y cuando alguna sea \(\bot\) y cuando la 
otra sea \(\top\). Y es por esa razón que cuando tenemos \(n\) variables 
proposicionales tenemos \(2^{n}\) estados posibles.
\subsubsection{Interpretación}
\noindent
Dado un estado de las variables $\mathcal{I} \ : \ VarP \to \text{\texttt{Bool}}$, 
definimos la interpretación de las fórmulas con respecto a $\mathcal{I}$ como la función
$\mathcal{I}^{\star} \ : \ \text{\texttt{PROP}} \to \text{\texttt{Bool}}$ tal que:
\begin{itemize}
    \item $\mathcal{I}^{\star}(p)=\mathcal{I}(p)$ para $p\in VarP$, es decir
    \(\mathcal{I}^{\star}|_{VarP} = \mathcal{I}\)
    %$\restr{\mathcal{I}^{\star}}{_{VarP}}=\mathcal{I}$.
    \item $\mathcal{I}^{\star}(\top)=1$
    \item $\mathcal{I}^{\star}(\bot)=0$
    \item $\mathcal{I}^{\star}(\lnot\varphi)=1$ sí y solamente sí $\mathcal{I}^{\star}(\varphi)=0$.
    \item $\mathcal{I}^{\star}(\varphi\land\psi)=1$ sí y solamente sí $\mathcal{I}^{\star}(\varphi)=\mathcal{I}^{\star}(\psi)=1$.
    \item $\mathcal{I}^{\star}(\varphi\lor\psi)=0$ sí y solamente sí $\mathcal{I}^{\star}(\varphi)=\mathcal{I}^{\star}(\psi)=0$.
    \item $\mathcal{I}^{\star}(\varphi \rightarrow \psi)=0$ sí y solamente sí $\mathcal{I}^{\star}(\varphi)=1$ \textbf{e}
      $\mathcal{I}^{\star}(\psi)=0$.
    \item $\mathcal{I}^{\star}(\varphi \leftrightarrow \psi)=1$ sí y solamente sí $\mathcal{I}^{\star}(\varphi)=\mathcal{I}^{\star}(\psi)$. 
\end{itemize}
\subsubsection{Sobrecarga de operadores}
Obsérvese que dado un estado de las variables $\mathcal{I}$, %(o tambi\'en~$\I_v$), 
la interpretación~$\mathcal{I}^{\star}$ generada por~$\mathcal{I}$ est\'a determinada de manera 
\'unica por lo que de ahora en adelante escribiremos simplemente $\mathcal{I}$ en lugar 
de~$\mathcal{I}^{\star}$.
\subsection{Lema de coincidencia}
\noindent
Sean $\mathcal{I}_1,\mathcal{I}_2 \ : \ VarP \to \text{\texttt{Bool}}$ dos estados que coinciden en las variables 
proposicionales de la fórmula $\varphi$, es decir 
$\mathcal{I}_{1}(p)=\mathcal{I}_2(p)$ para toda $p\in vars(\varphi)$.\\
Entonces $\mathcal{I}_1(\varphi)=\mathcal{I}_2(\varphi)$.
\begin{proof}
    La demostración se hará usando el \textbf{Principio de Inducción Estructural}
    \newline
    \textbf{Caso Base}
    \newline
    Sea \(\varphi \in \text{\texttt{ATOM}}\), notemos que solamente tenemos un caso ya que la hipótesis sólo 
    aplica para variables proposicionales y por lo tanto descartamos a las constantes lógicas. 
    Sean \(\mathcal{I}_{1}, \mathcal{I}_{2}: VarP \to \text{\texttt{Bool}}\) dos estados, por tales que
    $\mathcal{I}_{1}(p)=\mathcal{I}_2(p)$ para toda $p\in vars(\varphi)$, como \(\varphi\) es una variable
    proposicional por suposición, entonces se sigue que $\mathcal{I}_1(\varphi)=\mathcal{I}_2(\varphi)$.
    \newline
    \textbf{Hipótesis de Inducción}
    \newline
    Sean \(\gamma \in \text{\texttt{PROP}}\) y $\mathcal{I}_1,\mathcal{I}_2 \ : \ VarP \to \text{\texttt{Bool}}$
    dos estados tales que:
    \begin{itemize}
        \item $\mathcal{I}_{1}(p)=\mathcal{I}_2(p)$ para toda $p\in vars(\gamma)$
    \end{itemize}
    Entonces 
    \begin{itemize}
        \item $\mathcal{I}_1(\gamma)=\mathcal{I}_2(\gamma)$
    \end{itemize}
    \textbf{Paso Inductivo}
    \newline
    Caso (1)
    \newline
    Sea \(\gamma \equiv \neg \varphi\) para alguna \(\varphi \in \text{\texttt{PROP}}\). Sean $\mathcal{I}_1,\mathcal{I}_2 \ : \ VarP \to \text{\texttt{Bool}}$
    dos estados tales que:
    \begin{itemize}
        \item $\mathcal{I}_{1}(p)=\mathcal{I}_2(p)$ para toda $p\in vars(\gamma)$
    \end{itemize}
    Definimos a las interpretaciones de la fórmula \(\varphi\) con respecto a \(\mathcal{I}_{1}\) y 
    \(\mathcal{I}_{2}\) como \(\mathcal{I}^{\star}_{1}\) y como \(\mathcal{I}^{\star}_{2}\) 
    respectivamente. Entonces por definición de \textbf{interpretación} tenemos que:
    \begin{itemize}
        \item \(\mathcal{I}^{\star}_{1}\left(\varphi\right) = 0\) sí y sólo sí \(\mathcal{I}^{\star}_{1}\left(\neg \varphi\right) = 1\)
        \item \(\mathcal{I}^{\star}_{1}\left(\varphi\right) = 1\) sí y sólo sí \(\mathcal{I}^{\star}_{1}\left(\neg \varphi\right) = 0\)
        \item \(\mathcal{I}^{\star}_{2}\left(\varphi\right) = 0\) sí y sólo sí \(\mathcal{I}^{\star}_{2}\left(\neg \varphi\right) = 1\)
        \item \(\mathcal{I}^{\star}_{2}\left(\varphi\right) = 1\) sí y sólo sí \(\mathcal{I}^{\star}_{2}\left(\neg \varphi\right) = 0\)
    \end{itemize}
    Aplicando \textbf{hipótesis de inducción} tenemos que:
    \[
        \mathcal{I}_1(\varphi)=\mathcal{I}_2(\varphi)
    \]
    Por la observación que se realizó acerca de la \textbf{sobrecarga de operadores}. Sabemos que 
    \begin{itemize}
        \item \(\mathcal{I}^{\star}_{1}\left(\varphi\right) = \mathcal{I}_{1}\left(\varphi\right)\)
        \item \(\mathcal{I}^{\star}_{2}\left(\varphi\right) = \mathcal{I}_{2}\left(\varphi\right)\)
    \end{itemize}
    Así 
    \begin{itemize}
        \item \(\mathcal{I}^{\star}_{1}\left(\varphi\right) = 0 = \mathcal{I}_{1}\left(\varphi\right)\) sí y sólo sí \(\mathcal{I}^{\star}_{1}\left(\neg \varphi\right) = 1\)
        \item \(\mathcal{I}^{\star}_{1}\left(\varphi\right) = 1 = \mathcal{I}_{1}\left(\varphi\right)\) sí y sólo sí \(\mathcal{I}^{\star}_{1}\left(\neg \varphi\right) = 0\)
        \item \(\mathcal{I}^{\star}_{2}\left(\varphi\right) = 0 = \mathcal{I}_{2}\left(\varphi\right)\) sí y sólo sí \(\mathcal{I}^{\star}_{2}\left(\neg \varphi\right) = 1\)
        \item \(\mathcal{I}^{\star}_{2}\left(\varphi\right) = 1 = \mathcal{I}_{2}\left(\varphi\right)\) sí y sólo sí \(\mathcal{I}^{\star}_{2}\left(\neg \varphi\right) = 0\)
    \end{itemize}
    Independientemente del valor de \(\neg \varphi\) y aplicando la observación hecha para la \textbf{sobrecarga de operadores} tenemos que:
    \[
        \mathcal{I}_{1}\left(\neg \varphi\right) = \mathcal{I}^{\star}_{1}\left(\neg \varphi\right) = \mathcal{I}^{\star}_{2}\left(\neg \varphi\right) = \mathcal{I}_{2}\left(\neg \varphi\right)
    \]
    Caso (2)
    \newline
    Sea \(\gamma \equiv \varphi \star \psi \) con \(\varphi, \psi \in \text{\texttt{PROP}}\) y \(\star \in \{\land, \lor, \to, \leftrightarrow\}\). Sean $\mathcal{I}_1,\mathcal{I}_2 \ : \ VarP \to \text{\texttt{Bool}}$
    dos estados tales que:
    \begin{itemize}
        \item $\mathcal{I}_{1}(p)=\mathcal{I}_2(p)$ para toda $p\in vars(\varphi)$
        \item $\mathcal{I}_{1}(q)=\mathcal{I}_2(q)$ para toda $q\in vars(\psi)$
    \end{itemize}
    Definimos a las interpretaciones de la fórmula \(\varphi\) con respecto a \(\mathcal{I}_{1}\) y 
    \(\mathcal{I}_{2}\) como \(\mathcal{I}^{\star}_{1}\) y como \(\mathcal{I}^{\star}_{2}\) 
    respectivamente.
    \newline 
    Conjunción (\(\land\))
    \newline
    Entonces por definición de \textbf{interpretación} tenemos que:
    \begin{itemize}
        \item \(\mathcal{I}^{\star}_{1}\left(\varphi \land \psi\right) = 1\) sí y sólo sí \(\mathcal{I}^{\star}_{1}\left( \varphi\right) = \mathcal{I}^{\star}_{1}\left( \psi\right) = 1\)
        \item \(\mathcal{I}^{\star}_{2}\left(\varphi \land \psi\right) = 1\) sí y sólo sí \(\mathcal{I}^{\star}_{2}\left( \varphi\right) = \mathcal{I}^{\star}_{2}\left( \psi\right) = 1\)
        \item \(\mathcal{I}^{\star}_{1}\left(\varphi \land \psi\right) = 0\) sí y sólo sí \(\mathcal{I}^{\star}_{1}\left( \varphi\right) = 0\) ó \(\mathcal{I}^{\star}_{1}\left( \psi\right) = 0\)
        \item \(\mathcal{I}^{\star}_{2}\left(\varphi \land \psi\right) = 0\) sí y sólo sí \(\mathcal{I}^{\star}_{2}\left( \varphi\right) = 0\) ó \(\mathcal{I}^{\star}_{2}\left( \psi\right) = 0\)
    \end{itemize}
    Aplicando \textbf{hipótesis de inducción} tenemos que:
    \begin{align*}
        \mathcal{I}_1(\varphi) &= \mathcal{I}_2(\varphi) \\
        \mathcal{I}_1(\psi) &= \mathcal{I}_2(\psi)
    \end{align*}
    Por la observación que se realizó acerca de la \textbf{sobrecarga de operadores}. Sabemos que 
    \begin{itemize}
        \item \(\mathcal{I}^{\star}_{1}\left(\varphi\right) = \mathcal{I}_{1}\left(\varphi\right)\)
        \item \(\mathcal{I}^{\star}_{2}\left(\varphi\right) = \mathcal{I}_{2}\left(\varphi\right)\)
    \end{itemize}
    Así
    \begin{itemize}
        \item \(\mathcal{I}^{\star}_{1}\left(\varphi \land \psi\right) = 1\) sí y sólo sí \(\mathcal{I}^{\star}_{1}\left( \varphi\right) = \mathcal{I}_{1}\left( \varphi\right) = \mathcal{I}^{\star}_{1}\left( \psi\right) = \mathcal{I}_{1}\left( \psi\right) = 1\)
        \item \(\mathcal{I}^{\star}_{2}\left(\varphi \land \psi\right) = 1\) sí y sólo sí \(\mathcal{I}^{\star}_{2}\left( \varphi\right) = \mathcal{I}_{2}\left( \varphi\right) =  \mathcal{I}^{\star}_{2}\left( \psi\right) = \mathcal{I}_{2}\left( \psi\right) = 1\)
        \item \(\mathcal{I}^{\star}_{1}\left(\varphi \land \psi\right) = 0\) sí y sólo sí \(\mathcal{I}^{\star}_{1}\left( \varphi\right) = \mathcal{I}_{1}\left( \varphi\right) = 0\) ó \(\mathcal{I}^{\star}_{1}\left( \psi\right) = \mathcal{I}_{1}\left( \psi \right) = 0\)
        \item \(\mathcal{I}^{\star}_{2}\left(\varphi \land \psi\right) = 0\) sí y sólo sí \(\mathcal{I}^{\star}_{2}\left( \varphi\right) = \mathcal{I}_{2}\left( \varphi\right) = 0\) ó \(\mathcal{I}^{\star}_{2}\left( \psi\right) = \mathcal{I}_{2}\left( \psi \right) = 0\) 
    \end{itemize}
    Independientemente del valor de \(\varphi \land \psi\) y aplicando la observación hecha para la \textbf{sobrecarga de operadores} tenemos que:
    \[
        \mathcal{I}_{1}\left(\varphi \land \psi\right) = \mathcal{I}^{\star}_{1}\left(\varphi \land \psi\right) = \mathcal{I}^{\star}_{2}\left(\varphi \land \psi\right) = \mathcal{I}_{2}\left(\varphi \land \psi\right)
    \] 
    Disyunción (\(\lor\))
    \newline
    Entonces por definición de \textbf{interpretación} tenemos que:
    \begin{itemize}
        \item \(\mathcal{I}^{\star}_{1}\left(\varphi \lor \psi\right) = 0\) sí y sólo sí \(\mathcal{I}^{\star}_{1}\left( \varphi\right) = \mathcal{I}^{\star}_{1}\left( \psi\right) = 0\)
        \item \(\mathcal{I}^{\star}_{2}\left(\varphi \lor \psi\right) = 0\) sí y sólo sí \(\mathcal{I}^{\star}_{2}\left( \varphi\right) = \mathcal{I}^{\star}_{2}\left( \psi\right) = 0\)
        \item \(\mathcal{I}^{\star}_{1}\left(\varphi \lor \psi\right) = 1\) sí y sólo sí \(\mathcal{I}^{\star}_{1}\left( \varphi\right) = 1\) ó \(\mathcal{I}^{\star}_{1}\left( \psi\right) = 1\)
        \item \(\mathcal{I}^{\star}_{2}\left(\varphi \lor \psi\right) = 1\) sí y sólo sí \(\mathcal{I}^{\star}_{2}\left( \varphi\right) = 1\) ó \(\mathcal{I}^{\star}_{2}\left( \psi\right) = 1\)
    \end{itemize}
    Aplicando \textbf{hipótesis de inducción} tenemos que:
    \begin{align*}
        \mathcal{I}_1(\varphi) &= \mathcal{I}_2(\varphi) \\
        \mathcal{I}_1(\psi) &= \mathcal{I}_2(\psi)
    \end{align*}
    Por la observación que se realizó acerca de la \textbf{sobrecarga de operadores}. Sabemos que 
    \begin{itemize}
        \item \(\mathcal{I}^{\star}_{1}\left(\varphi\right) = \mathcal{I}_{1}\left(\varphi\right)\)
        \item \(\mathcal{I}^{\star}_{2}\left(\varphi\right) = \mathcal{I}_{2}\left(\varphi\right)\)
    \end{itemize}
    Así
    \begin{itemize}
        \item \(\mathcal{I}^{\star}_{1}\left(\varphi \lor \psi\right) = \mathcal{I}_{1}\left(\varphi \lor \psi\right) = 0\) sí y sólo sí \(\mathcal{I}^{\star}_{1}\left( \varphi\right) = \mathcal{I}_{1}\left( \varphi\right) = \mathcal{I}^{\star}_{1}\left( \psi\right) = \mathcal{I}_{1}\left( \psi\right) = 0\)
        \item \(\mathcal{I}^{\star}_{2}\left(\varphi \lor \psi\right) = \mathcal{I}_{2}\left(\varphi \lor \psi\right) = 0\) sí y sólo sí \(\mathcal{I}^{\star}_{2}\left( \varphi\right) = \mathcal{I}_{2}\left( \varphi\right) = \mathcal{I}^{\star}_{2}\left( \psi\right) = \mathcal{I}_{2}\left( \psi\right) = 0\)
        \item \(\mathcal{I}^{\star}_{1}\left(\varphi \lor \psi\right) = \mathcal{I}_{1}\left(\varphi \lor \psi\right) = 1\) sí y sólo sí \(\mathcal{I}^{\star}_{1}\left( \varphi\right) = \mathcal{I}_{1}\left( \varphi\right) = 1\) ó \(\mathcal{I}^{\star}_{1}\left(\psi\right) = \mathcal{I}_{1}\left(\psi\right) = 1\)
        \item \(\mathcal{I}^{\star}_{2}\left(\varphi \lor \psi\right) = \mathcal{I}_{2}\left(\varphi \lor \psi\right) = 1\) sí y sólo sí \(\mathcal{I}^{\star}_{2}\left( \varphi\right) = \mathcal{I}_{2}\left( \varphi\right) = 1\) ó \(\mathcal{I}^{\star}_{2}\left(\psi\right) = \mathcal{I}_{2}\left(\psi\right) = 1\)
    \end{itemize}
    Independientemente del valor de \(\varphi \lor \psi\) y aplicando la observación hecha para la \textbf{sobrecarga de operadores} tenemos que:
    \[
        \mathcal{I}_{1}\left(\varphi \lor \psi\right) = \mathcal{I}^{\star}_{1}\left(\varphi \lor \psi\right) = \mathcal{I}^{\star}_{2}\left(\varphi \lor \psi\right) = \mathcal{I}_{2}\left(\varphi \lor \psi\right)
    \]
    Implicación (\(\to\))
    \newline
    Entonces por definición de \textbf{interpretación} tenemos que:
    \begin{itemize}
        \item \(\mathcal{I}^{\star}_{1}\left(\varphi \to \psi\right) = 0\) sí y sólo sí \(\mathcal{I}^{\star}_{1}\left( \varphi\right) = 1\) e \(\mathcal{I}^{\star}_{1}\left( \psi\right) = 0\)
        \item \(\mathcal{I}^{\star}_{2}\left(\varphi \to \psi\right) = 0\) sí y sólo sí \(\mathcal{I}^{\star}_{2}\left( \varphi\right) = 1\) e \(\mathcal{I}^{\star}_{2}\left( \psi\right) = 0\)
        \item \(\mathcal{I}^{\star}_{1}\left(\varphi \to \psi\right) = 1\) sí y sólo sí \(\mathcal{I}^{\star}_{1}\left( \varphi\right) = \mathcal{I}^{\star}_{1}\left(\psi\right) = 1\) ó 
        \(\mathcal{I}^{\star}_{1}\left(\varphi\right) = \mathcal{I}^{\star}_{1}\left(\psi\right) = 0\) ó 
        \(\mathcal{I}^{\star}_{2}\left( \varphi\right) = 0\) e \(\mathcal{I}^{\star}_{2}\left( \psi\right) = 1\)
        \item \(\mathcal{I}^{\star}_{2}\left(\varphi \to \psi\right) = 1\) sí y sólo sí \(\mathcal{I}^{\star}_{2}\left( \varphi\right) = \mathcal{I}^{\star}_{2}\left(\psi\right) = 1\) ó 
        \(\mathcal{I}^{\star}_{2}\left(\varphi\right) = \mathcal{I}^{\star}_{2}\left(\psi\right) = 0\) ó 
        \(\mathcal{I}^{\star}_{2}\left( \varphi\right) = 0\) e \(\mathcal{I}^{\star}_{2}\left( \psi\right) = 1\)
    \end{itemize}
    Aplicando \textbf{hipótesis de inducción} tenemos que:
    \begin{align*}
        \mathcal{I}_1(\varphi) &= \mathcal{I}_2(\varphi) \\
        \mathcal{I}_1(\psi) &= \mathcal{I}_2(\psi)
    \end{align*}
    Por la observación que se realizó acerca de la \textbf{sobrecarga de operadores}. Sabemos que 
    \begin{itemize}
        \item \(\mathcal{I}^{\star}_{1}\left(\varphi\right) = \mathcal{I}_{1}\left(\varphi\right)\)
        \item \(\mathcal{I}^{\star}_{2}\left(\varphi\right) = \mathcal{I}_{2}\left(\varphi\right)\)
    \end{itemize}
    Así
    \begin{itemize}
        \item \(\mathcal{I}^{\star}_{1}\left(\varphi \to \psi\right) = 0\) sí y sólo sí \(\mathcal{I}^{\star}_{1}\left( \varphi\right) = \mathcal{I}_{1}\left( \varphi\right) = 1\) e \(\mathcal{I}^{\star}_{1}\left( \psi\right) = \mathcal{I}_{1}\left( \psi\right) = 0\)
        \item \(\mathcal{I}^{\star}_{2}\left(\varphi \to \psi\right) = 0\) sí y sólo sí \(\mathcal{I}^{\star}_{2}\left( \varphi\right) = \mathcal{I}_{2}\left( \varphi\right) = 1\) e \(\mathcal{I}^{\star}_{2}\left( \psi\right) = \mathcal{I}_{2}\left( \psi\right) = 0\)
        \item \(\mathcal{I}^{\star}_{1}\left(\varphi \to \psi\right) = 1\) sí y sólo sí 
        \(\mathcal{I}^{\star}_{1}\left( \varphi\right) = \mathcal{I}_{1}\left( \varphi\right) = \mathcal{I}^{\star}_{1}\left(\psi\right) = \mathcal{I}_{1}\left(\psi\right) = 1\) ó 
        \newline
        \(\mathcal{I}^{\star}_{1}\left(\varphi\right) = \mathcal{I}_{1}\left(\varphi\right) = \mathcal{I}^{\star}_{1}\left(\psi\right) = \mathcal{I}_{1}\left(\psi\right) = 0\) ó 
        \newline
        \(\mathcal{I}^{\star}_{2}\left( \varphi\right) = \mathcal{I}_{2}\left( \varphi\right) = 0\) e \(\mathcal{I}^{\star}_{2}\left( \psi\right) = \mathcal{I}_{2}\left( \psi\right) = 1\)
        \item \(\mathcal{I}^{\star}_{2}\left(\varphi \to \psi\right) = 1\) sí y sólo sí 
        \(\mathcal{I}^{\star}_{2}\left( \varphi\right) = \mathcal{I}_{2}\left( \varphi\right) = \mathcal{I}^{\star}_{2}\left(\psi\right) = \mathcal{I}_{2}\left(\psi\right) = 1\) ó 
        \newline
        \(\mathcal{I}^{\star}_{2}\left(\varphi\right) = \mathcal{I}_{2}\left(\varphi\right) = \mathcal{I}^{\star}_{2}\left(\psi\right) = \mathcal{I}_{2}\left(\psi\right) = 0\) ó 
        \newline
        \(\mathcal{I}^{\star}_{2}\left( \varphi\right) = \mathcal{I}_{2}\left( \varphi\right) = 0\) e \(\mathcal{I}^{\star}_{2}\left( \psi\right) = \mathcal{I}_{2}\left( \psi\right) = 1\)
    \end{itemize}
    Independientemente del valor de \(\varphi \to \psi\) y aplicando la observación hecha para la \textbf{sobrecarga de operadores} tenemos que:
    \[
        \mathcal{I}_{1}\left(\varphi \to \psi\right) = \mathcal{I}^{\star}_{1}\left(\varphi \to \psi\right) = \mathcal{I}^{\star}_{2}\left(\varphi \to \psi\right) = \mathcal{I}_{2}\left(\varphi \to \psi\right)
    \]
    Doble Implicación (\(\leftrightarrow\))
    \newline
    Entonces por definición de \textbf{interpretación} tenemos que:
    \begin{itemize}
        \item \(\mathcal{I}^{\star}_{1}\left(\varphi \leftrightarrow \psi\right) = 0\) sí y sólo sí \(\mathcal{I}^{\star}_{1}\left( \varphi\right) \neq \mathcal{I}^{\star}_{1}\left( \psi\right)\)
        \item \(\mathcal{I}^{\star}_{2}\left(\varphi \leftrightarrow \psi\right) = 0\) sí y sólo sí \(\mathcal{I}^{\star}_{2}\left( \varphi\right) \neq \mathcal{I}^{\star}_{2}\left( \psi\right)\)
        \item \(\mathcal{I}^{\star}_{1}\left(\varphi \leftrightarrow \psi\right) = 1\) sí y sólo sí \(\mathcal{I}^{\star}_{1}\left( \varphi\right) = \mathcal{I}^{\star}_{1}\left( \psi\right)\)
        \item \(\mathcal{I}^{\star}_{2}\left(\varphi \leftrightarrow \psi\right) = 1\) sí y sólo sí \(\mathcal{I}^{\star}_{2}\left( \varphi\right) = \mathcal{I}^{\star}_{2}\left( \psi\right)\)
    \end{itemize}
    Aplicando \textbf{hipótesis de inducción} tenemos que:
    \begin{align*}
        \mathcal{I}_1(\varphi) &= \mathcal{I}_2(\varphi) \\
        \mathcal{I}_1(\psi) &= \mathcal{I}_2(\psi)
    \end{align*}
    Por la observación que se realizó acerca de la \textbf{sobrecarga de operadores}. Sabemos que 
    \begin{itemize}
        \item \(\mathcal{I}^{\star}_{1}\left(\varphi\right) = \mathcal{I}_{1}\left(\varphi\right)\)
        \item \(\mathcal{I}^{\star}_{2}\left(\varphi\right) = \mathcal{I}_{2}\left(\varphi\right)\)
    \end{itemize}
    Así
    \begin{itemize}
        \item \(\mathcal{I}^{\star}_{1}\left(\varphi \leftrightarrow \psi\right) = 0\) sí y sólo sí 
        \(\mathcal{I}_{1}\left( \varphi\right) = \mathcal{I}^{\star}_{1}\left( \varphi\right) \neq 
        \mathcal{I}^{\star}_{1}\left( \psi\right) = \mathcal{I}_{1}\left( \psi\right)\)
        \item \(\mathcal{I}^{\star}_{2}\left(\varphi \leftrightarrow \psi\right) = 0\) sí y sólo sí 
        \(\mathcal{I}_{2}\left( \varphi\right) = \mathcal{I}^{\star}_{2}\left( \varphi\right) \neq 
        \mathcal{I}^{\star}_{2}\left( \psi\right) = \mathcal{I}_{2}\left( \psi\right)\)
        \item \(\mathcal{I}^{\star}_{1}\left(\varphi \leftrightarrow \psi\right) = 1\) sí y sólo sí 
        \(\mathcal{I}^{\star}_{1}\left( \varphi\right) = \mathcal{I}_{1}\left( \varphi\right) 
        = \mathcal{I}^{\star}_{1}\left( \psi\right) = \mathcal{I}^{\star}\left( \psi\right)\)
        \item \(\mathcal{I}^{\star}_{2}\left(\varphi \leftrightarrow \psi\right) = 1\) sí y sólo sí 
        \(\mathcal{I}^{\star}_{2}\left( \varphi\right) = \mathcal{I}_{2}\left( \varphi\right) 
        = \mathcal{I}^{\star}_{2}\left( \psi\right) = \mathcal{I}^{\star}\left( \psi\right)\)
    \end{itemize}
    Independientemente del valor de \(\varphi \leftrightarrow \psi\) y aplicando la observación hecha para la \textbf{sobrecarga de operadores} tenemos que:
    \[
        \mathcal{I}_{1}\left(\varphi \to \psi\right) = \mathcal{I}^{\star}_{1}\left(\varphi \to \psi\right) = \mathcal{I}^{\star}_{2}\left(\varphi \to \psi\right) = \mathcal{I}_{2}\left(\varphi \to \psi\right)
    \]
    Por lo tanto, queda demostrado el lema.
\end{proof}
\subsection{Estado modificado o actualizado}
\noindent
Sean $\mathcal{I}: VarP \to \text{\texttt{Bool}}$ un estado de las variables, $p$ una variable 
proposicional y $v \in \text{\texttt{Bool}}$. Definimos la actualización de $\mathcal{I}$ en $p$ por 
$v$, denotado $\mathcal{I}[p/v]$ como sigue:
\[
    \mathcal{I}\left[p / v\right]\left(q\right) =
    \begin{cases}
        v & \text{si } q = p \\
        q & \text{si } q \neq p
    \end{cases}
\]
\subsection{Lema de sustitución}
Sean $\mathcal{I}$ una interpretación, $p$ una variable 
proposicional y $\psi$ una fórmula tal que $\mathcal{I}^{\star}(\psi) = v$. Entonces
\[
    \mathcal{I}\left(\varphi\left[p \ := \ \psi\right]\right) = \mathcal{I}\left[p / v\right]\left(\varphi\right)
\]
\section{Conceptos Semánticos Básicos}
\subsection{Tautología}
Si $\mathcal{I}(\varphi)=1$ para toda interpretación $\mathcal{I}$ decimos que 
$\varphi$ es una tautología o fórmula válida y escribimos $\models\varphi$.
\subsection{Satisfacible}
Si $\mathcal{I}(\varphi)=1$ para alguna interpretación $\mathcal{I}$ decimos que 
$\varphi$ es satisfacible,  que \(\varphi\) es verdadera en \(\mathcal{I}\) y 
escribimos $ \mathcal{I} \models\varphi$.
\subsection{Insatisfacible}
Si $\mathcal{I}(\varphi)=0$ para alguna interpretación $\mathcal{I}$ decimos que 
$\varphi$ es insatisfacible o es falsa en \(\mathcal{I}\) y 
escribimos $ \mathcal{I} \nvDash\varphi$.
\subsection{Contradicción}
Si $\mathcal{I}(\varphi)=0$ para toda interpretación $\mathcal{I}$ decimos que 
$\varphi$ es una contradicción o una fórmula no satisfacible.
\subsection{Conjunto de fórmulas}
Sea $\Gamma$ es un conjunto de fórmulas decimos que:
\begin{itemize}
    \item $\Gamma$ es satisfacible si tiene un modelo, es decir, si existe una 
    interpretación $\mathcal{I}$ tal que $\mathcal{I}(\varphi)=1$ para toda 
    $\varphi\in\Gamma$. Lo cual denotamos a veces, abusando de la notación, 
    con $\mathcal{I}(\Gamma)=1$.
    \item $\Gamma$ es insatisfacible o no satisfacible si no tiene un
    modelo, es decir, si no existe una interpretación $\mathcal{I}$ tal que 
    $\mathcal{I}(\varphi)=1$ para toda $\varphi\in\Gamma$.
\end{itemize}
\subsection{Proposición}
Sea $\Gamma$ un conjunto de fórmulas, $\varphi \in \Gamma$, $\tau$ una 
tautología y $\chi$ una contradicción. 
Si \(\Gamma\) es satisfacible entonces
\begin{enumerate}
    \item $\Gamma\setminus\{\varphi\}$ es satisfacible.
    \item $\Gamma\cup\{\tau\}$ es satisfacible.
    \item $\Gamma\cup\{\chi\}$ es insatisfacible.
\end{enumerate}
Si \(\Gamma\) es insatisfacible, con \(\tau \in \Gamma\) entonces
\begin{enumerate}
    \item $\Gamma\cup\{\psi\}$ es insatisfacible, para cualquier \(\psi \in\) \texttt{PROP}.
    \item $\Gamma\setminus\{\tau\}$ es insatisfacible.
\end{enumerate}
Veamos (1)
\begin{proof}
    Definimos a \(\Gamma\) como sigue:
    \[
        \Gamma = \{
            \psi_{1}, \psi_{2}, \dotsc, \psi_{n}, \varphi
        \}  
    \]
    Como \(\Gamma\) es satisfacible, entonces existe una interpretación \(\mathcal{I}\) 
    tal que \(\mathcal{I}\left(\kappa\right) = 1\) para toda \(\kappa \in \Gamma\).
    \newline 
    Sea \(\Gamma' = \Gamma \setminus \{\varphi\}\), entonces definimos a \(\mathcal{I}'\)
    como 
    \[
        \mathcal{I}'\left(\kappa\right) = \mathcal{I}\left(\kappa\right)
    \]
    Para toda \(\kappa \in \Gamma'\). Por definición de \(\mathcal{I}\) tenemos que
    \[
        \mathcal{I}'\left(\kappa\right) = 1
    \]
    Para toda \(\kappa \in \mathcal{I}'\). Por lo tanto \(\Gamma'\) es satisfacible
\end{proof}
Veamos (2)
\begin{proof}
    Definimos a \(\Gamma\) como sigue:
    \[
        \Gamma = \{
            \psi_{1}, \psi_{2}, \dotsc, \psi_{n}, \varphi
        \}  
    \]
    Como \(\Gamma\) es satisfacible, entonces existe una interpretación \(\mathcal{I}\) 
    tal que \(\mathcal{I}\left(\kappa\right) = 1\) para toda \(\kappa \in \Gamma\).
    \newline 
    Sea \(\Gamma' = \Gamma \cup \{\tau\}\), entonces
    como 
    \[
        \mathcal{I}\left(\tau\right) = 1
    \]
    ya que \(\tau\) es tautología. Por lo tanto \(\Gamma'\) es satisfacible bajo \(\mathcal{I}\).
\end{proof}
Veamos (3)
\newline
\begin{proof}
    Sea \(\Gamma' = \Gamma \cup \{\chi\}\), entonces
    como 
    \[
        \mathcal{I}\left(\chi\right) = 0
    \]
    para cualquier interpretación \(\mathcal{I}\), entonces el conjunto \(\Gamma'\) no es satisfacible.
    Ya que en particular \(\chi\) siempre se evaluará en \(0\).
\end{proof}
\newpage
\subsection{Proposición}
Sea $\Gamma=\{\varphi_1,\ldots,\varphi_n\}$ un conjunto de fórmulas. 
\begin{enumerate}
    \item $\Gamma$ es satisfacible sí y sólo sí $\varphi_1\land\ldots\land\varphi_n$ es
    satisfacible.
    \item $\Gamma$ es insatisfacible sí y sólo sí $\varphi_1\land\ldots\land\varphi_n$ es
    una contradicción.
\end{enumerate}
Veamos (1)
\newline 
\begin{proof}
    (Ida)
    \newline
    Como \(\Gamma\) es satisfacible, entonces existe una interpretación \(\mathcal{I}\)
    tal que para toda \(\varphi_{i} \in \Gamma\) con \(1 \leq i \leq n\). Entonces tenemos que 
    se cumple que:
    \begin{itemize}
       \item \(\mathcal{I}\left(\varphi_{1}\right) = 1\) y
       \item \(\mathcal{I}\left(\varphi_{2}\right) = 1\) y
       \item ...
       \item \(\mathcal{I}\left(\varphi_{n}\right) = 1\).
    \end{itemize}
    Entonces tenemos que 
    \[
        \mathcal{I}\left(\varphi_{1}\right) = \mathcal{I}\left(\varphi_{2}\right) = \dotsc = \mathcal{I}\left(\varphi_{n}\right) = 1
    \]
    Consecuentemente
    \[  
        \mathcal{I}\left(\varphi_{1} \land \varphi_{2} \land \dotsc \land \varphi_{n}\right) = 1  
    \]
    Por lo tanto 
    \[
        \varphi_{1} \land \varphi_{2} \land \dotsc \land \varphi_{n}  
    \]
    es satisfacible bajo \(\mathcal{I}\).
    \newline
    (Vuelta)
    \newline
    Como
    \[
        \varphi_{1} \land \varphi_{2} \land \dotsc \land \varphi_{n}  
    \]
    es satisfacible bajo \(\mathcal{I}\). Entonces
    \[  
        \mathcal{I}\left(\varphi_{1} \land \varphi_{2} \land \dotsc \land \varphi_{n}\right) = 1  
    \]
    que implica que
    \[
        \mathcal{I}\left(\varphi_{1}\right) = \mathcal{I}\left(\varphi_{2}\right) = \dotsc = \mathcal{I}\left(\varphi_{n}\right) = 1
    \]
    lo anterior es equivalente a lo siguiente:
    \begin{itemize}
        \item \(\mathcal{I}\left(\varphi_{1}\right) = 1\) y
        \item \(\mathcal{I}\left(\varphi_{2}\right) = 1\) y
        \item ...
        \item \(\mathcal{I}\left(\varphi_{n}\right) = 1\).
     \end{itemize}
    Por lo tanto tenemos que \(\Gamma\) es satisfacible.
\end{proof}
\section{Ejercicios}
\subsection{Ejercicio 1}
Defina utilizando los conectivos lógicos vistos en clase el operador
$\oplus$ (ó exclusivo), cuya propiedad es:
\[
    \mathcal{I}\left(\varphi \oplus \psi\right) = 1
\]
sí y sólo sí
\[
    \mathcal{I}\left(\varphi\right) \neq \mathcal{I}\left(\psi\right)
\]
\subsubsection{Solución}
Sabemos que 
\begin{align*}
    \mathcal{I}\left(\varphi \leftrightarrow \psi\right) &= 1 &\text{sí y sólo sí} \\
    \mathcal{I}\left(\varphi\right) &= \mathcal{I}\left(\varphi\right)
\end{align*}
Entonces por \textbf{contra-positiva} tenemos que
\begin{align*}
    \mathcal{I}\left(\varphi \leftrightarrow \psi\right) &= 0 &\text{sí y sólo sí} \\
    \mathcal{I}\left(\varphi\right) &\neq \mathcal{I}\left(\varphi\right)
\end{align*}
Sabemos que 
\begin{align*}
    \mathcal{I}\left(\gamma\right) &= 0 &\text{sí y sólo sí} \\
    \mathcal{I}\left(\neg \gamma\right) &=1 &
\end{align*}
Consecuentemente
\begin{align*}
    \mathcal{I}\left(\varphi \leftrightarrow \psi\right) &= 0 &\text{sí y sólo sí} \\
    \mathcal{I}\left(\neg \left(\varphi \leftrightarrow \psi\right)\right) &= 1
\end{align*}
Por lo tanto
\begin{align*}
    \mathcal{I}\left(\neg \left(\varphi \leftrightarrow \psi\right)\right) &= 1 &\text{sí y sólo sí} \\
    \mathcal{I}\left(\varphi\right) &\neq \mathcal{I}\left(\varphi\right)    
\end{align*}
Podemos concluir que:
\begin{align*}
    \neg \left(\varphi \leftrightarrow \psi\right) \equiv \varphi \oplus \psi
\end{align*}
Por equivalencias lógicas
\begin{align*}
    \neg \left(\varphi \leftrightarrow \psi\right) &\equiv \neg \left(\left(\varphi \rightarrow \psi\right) \land \left( \psi \rightarrow \varphi  \right)\right) &\text{Eliminación de \(\leftrightarrow\)} \\
                                                   &\equiv  \neg \left(\left(\neg \varphi \lor \psi\right) \land \left( \neg \psi \lor \varphi  \right)\right) &\text{Eliminación de \(\rightarrow\)} \\
                                                   &\equiv  \neg \left(\neg \varphi \lor \psi\right) \lor \neg \left( \neg \psi \lor \varphi  \right) &\text{De Morgan} \\
                                                   &\equiv   \left(\neg\neg \varphi \land \neg \psi\right) \lor \left( \neg \neg \psi \land \neg \varphi  \right) &\text{De Morgan} \\
                                                   &\equiv   \left(\varphi \land \neg \psi\right) \lor \left( \psi \land \neg \varphi  \right) &\text{Doble Negación} \\
                                                   &\equiv   \left(\varphi \land \neg \psi\right) \lor \left( \neg \varphi \land \psi  \right) &\text{Conmutatividad} \\
\end{align*}
Tenemos que
\[
    \varphi \oplus \psi \equiv \neg \left(\varphi \leftrightarrow \psi\right) \equiv   \left(\varphi \land \neg \psi\right) \lor \left( \neg \varphi \land \psi  \right)
\]
Por transitividad podemos concluir que
\[
    \varphi \oplus \psi \equiv \left(\varphi \land \neg \psi\right) \lor \left( \neg \varphi \land \psi  \right)
\]
\subsubsection{Tabla de verdad}
\begin{center}
    \begin{tabular}{ccc}
    $\varphi$ & \(\psi\) & $\varphi \oplus \psi$\\
    \midrule
    1 & 1 & 0 \\
    1 & 0 & 1 \\
    0 & 1 & 1 \\
    0 & 0 & 0 \\
    \end{tabular}
\end{center}
\subsection{Ejercicio 2}
Demuestre que si \(\Gamma\) es insatisfacible, con \(\tau \in \Gamma\) entonces
\begin{enumerate}
    \item $\Gamma\cup\{\psi\}$ es insatisfacible, para cualquier \(\psi \in\) \texttt{PROP}.
\end{enumerate}
\(\tau\) es una \textbf{tautología}.
\begin{proof}
    Sea \(\Gamma' = \Gamma \cup \{\psi\}\), entonces
    como 
    \[
        \mathcal{I}\left(\chi\right) = 0
    \]
    para alguna \(\chi \in \Gamma\), entonces al agregar \(\psi\), seguirá sin existir una interpretación
    \(\mathcal{I}\) que haga verdadera tanto a \(\chi\) como a \(\psi\), por lo tanto \(\Gamma'\) es 
    insatisfacible.
\end{proof}
\subsection{Ejercicio 3}
\noindent
Sea $\Gamma=\{\varphi_1,\ldots,\varphi_n\}$ un conjunto de fórmulas. Demuestra $\Gamma$ es insatisfacible sí y sólo sí $\varphi_1\land\ldots\land\varphi_n$ es
una contradicción.
\begin{proof}
    Se sigue de la proposición anterior.
\end{proof}
\subsection{Ejercicio 4}
Decida si los siguientes conjuntos de proposiciones son satisfacibles por
medio de interpretaciones:
\begin{enumerate}
    \item $\Gamma = \{ p \to q,\; (s \lor p) \land \lnot q,\;\lnot s\}$
    \item $\Gamma = \{(p \to r) \lor (\lnot s \land p),\; s \to \lnot(p \land r),\; r \lor
          \lnot s\}$
    \item $\Gamma = \{p\lor (q\land s),\; (\lnot r \lor s) \land (s\to t),\; \lnot p \lor
          \lnot t \} $
\end{enumerate}
\subsubsection{Solución}
\noindent
Recordemos la siguiente equivalencia:
\newline
$\Gamma$ es satisfacible sí y sólo sí $\varphi_1\land\ldots\land\varphi_n$ es
satisfacible.
\newline 
A su vez sabemos que \(\Gamma\) es satisfacible si tiene un modelo, es decir, si existe una 
interpretaci\'on tal que $\mathcal{I}(\varphi)=1$ para toda 
$\varphi\in\Gamma$.
\newline 
Veamos (1)
\newline
Supongamos que \(\mathcal{I}\left(\Gamma\right) = 1\), lo anterior implica que
\begin{align*}
    \mathcal{I} \left(p \to q \right)  &= 1 \\
    \mathcal{I} \left((s \lor p) \land \lnot q \right) &= 1 \\
    \mathcal{I} \left(\lnot s\right)  &= 1
\end{align*}
Sabemos que \(\mathcal{I} \left(\lnot s\right) = 1\) sí y sólo sí
\[
    \mathcal{I} \left(s\right) = 0
\]
Sabemos que \(\mathcal{I} \left((s \lor p) \land \lnot q \right) = 1\) sí y sólo sí
\[
    \mathcal{I} \left(s \lor p \right) =  \mathcal{I} \left(\lnot q \right) = 1
\]
Sabemos que \(\mathcal{I} \left(\lnot q \right) = 1\) sí y sólo sí
\[
    \mathcal{I} \left( q \right) = 0
\]
Sabemos que \(\mathcal{I} \left(s \lor p \right) = 1\) sí y sólo sí
\[
    \mathcal{I} \left(s \right) = 1 \ \text{ó} \ \mathcal{I} \left( p \right) = 1
\]
Sin embargo como \(\mathcal{I} \left(s\right) = 0\) entonces tenemos que
\[
    \mathcal{I} \left( p \right) = 1
\]
Sabemos que \(\mathcal{I} \left(p \to q \right) = 0\) sí y sólo sí
\begin{align*}
    \mathcal{I} \left(p\right) &= 1 \\
    \mathcal{I} \left(q\right) &= 0
\end{align*}
Que justamente son las interpretaciones que obtuvimos. Por lo tanto el conjunto \(\Gamma\) no 
es satisfacible. Notemos que mostramos que para cualquier estado \textit{(para aquellos que hace sentido tomar en cuenta
por ejemplo no contamos que \(\mathcal{I}\left(s\right) = 1\) ya que automáticamente nos hace que \(\Gamma\) no 
es satisfacible)} de las variables no se satisface a \(\Gamma\).
\newline
Veamos (2)
\newline
Supongamos que \(\mathcal{I}\left(\Gamma\right) = 1\), lo anterior implica que
\begin{align*}
    \mathcal{I}\left((p \to r) \lor (\lnot s \land p)\right) &= 1 \\ 
    \mathcal{I}\left(s \to \lnot(p \land r)\right) &= 1 \\
    \mathcal{I}\left(r \lor \lnot s\right) &= 1
\end{align*}
Sabemos que \(\mathcal{I}\left(r \lor \lnot s\right) = 1\) sí y sólo sí
\[
    \mathcal{I} \left(r \right) = 1 \ \text{ó} \ \mathcal{I} \left( \neg s \right) = 1
\]
Supongamos que \(\mathcal{I} \left(r \right) = \mathcal{I} \left( \neg s \right) = 1\).
Sabemos que \(\mathcal{I} \left( \neg s \right) = 1\) sí y sólo sí
\[
    \mathcal{I} \left( s \right) = 0
\]
Por lo anterior sabemos que \(\mathcal{I}\left(\lnot(p \land r)\right) = 1\) ó
\(\mathcal{I}\left(\lnot(p \land r)\right) = 0\), supongamos que se cumple lo último.
Sabemos que \(\mathcal{I}\left(\lnot(p \land r)\right) = 0\) sí y sólo sí
\[
    \mathcal{I}\left(p \land r\right) = 1
\]
Sabemos que \(\mathcal{I}\left(p \land r\right) = 1\) sí y sólo sí 
\[
    \mathcal{I}\left(p\right) = \mathcal{I}\left(r\right) = 1
\]
Notemos que no hemos entrado en ninguna contradicción, ya que anteriormente supusimos que
\(\mathcal{I}\left(r\right) = 1\). Observemos que \(\mathcal{I}\left(p \to r\right) = 1\), 
ya que
\begin{align*}
    \mathcal{I}\left(p\right) &= 1 \\
    \mathcal{I}\left(r\right) &= 1
\end{align*}
Observemos que \(\mathcal{I}\left(\lnot s \land p\right) = 1\), 
ya que
\begin{align*}
    \mathcal{I}\left(p\right) &= 1 \\
    \mathcal{I}\left(\lnot s\right) &= 1
\end{align*}
Consecuentemente \(\mathcal{I}\left((p \to r) \lor (\lnot s \land p)\right) = 1\).
Por lo tanto \(\Gamma\) es satisfacible bajo el siguiente modelo:
\begin{align*}
    \mathcal{I}\left(p\right) &= 1 \\
    \mathcal{I}\left(r\right) &= 1 \\
    \mathcal{I}\left(s\right) &= 0
\end{align*}
Veamos (3)
\newline
Supongamos que \(\mathcal{I}\left(\Gamma\right) = 1\), lo anterior implica que
\begin{align*}
    \mathcal{I}\left(p \lor (q\land s) \right) &= 1 \\
    \mathcal{I}\left((\lnot r \lor s) \land (s\to t)\right) &= 1 \\
    \mathcal{I}\left(\lnot p \lor \lnot t\right) &= 1
\end{align*}
Sabemos que \(\mathcal{I}\left((\lnot r \lor s) \land (s\to t) \right) = 1\) sí y sólo sí
\[
    \mathcal{I}\left(\lnot r \lor s\right) = \mathcal{I}\left(s\to t \right) = 1
\]
Sabemos que \(\mathcal{I}\left(\lnot r \lor s\right) = 1\) sí y sólo sí
\[
    \mathcal{I}\left(\lnot r\right) = 1 \ \text{ó} \ \mathcal{I}\left(s\right) = 1
\]
Supongamos que \(\mathcal{I}\left(\lnot r\right) = \mathcal{I}\left(s\right) = 1\).
Sabemos que \(\mathcal{I}\left(\lnot r\right) = 1\) sí y sólo sí
\[
    \mathcal{I}\left(r\right) = 0
\]
Por suposición, tenemos que \(\mathcal{I}\left(t\right) = 1\) ya que
\begin{align*}
    \mathcal{I}\left(t\right) &= 1 \\
    \mathcal{I}\left(s\to t \right) &= 1
\end{align*}
Sabemos que \(\mathcal{I}\left(t\right) = 1\) sí y sólo sí
\[
    \mathcal{I}\left(\neg t\right) = 0
\]
Como tenemos que 
\[
    \mathcal{I}\left(\lnot p \lor \lnot t\right) = 1
\]
Entonces
\[
    \mathcal{I}\left(\lnot p\right) = 1
\]
Lo anterior ocurre sí y sólo sí
\[
    \mathcal{I}\left(p\right) = 0
\]
Notemos que \(\mathcal{I}\left(q \land s\right) = 1\), ya que
\[
    \mathcal{I}\left(p\right) = 0
\]
Sabemos que \(\mathcal{I}\left(q \land s\right) = 1\) sí y sólo sí
\[
    \mathcal{I}\left(q\right) = \mathcal{I}\left(s\right) = 1
\]
Por lo tanto \(\Gamma\) es satisfacible bajo el siguiente modelo:
\begin{align*}
    \mathcal{I}\left(q\right) &= 1 \\
    \mathcal{I}\left(s\right) &= 1 \\
    \mathcal{I}\left(p\right) &= 0 \\
    \mathcal{I}\left(t\right) &= 1 \\
    \mathcal{I}\left(r\right) &= 0
\end{align*}
\subsection{Ejercicio 5}
Pruebe la correctud de los siguientes argumentos de la l\'ogica
proposicional. Resuelva por interpretaciones apoy\'andose del principio de
refutaci\'on.
\begin{enumerate}
\item $p \lor q, \lnot r \to \lnot p$ entonces $q \to \lnot r$
\item $p \to q \lor r, r \to \lnot p, q \to \lnot p$ entonces $\lnot (p
\land \lnot s)$
\end{enumerate}
\subsubsection{Solución}
Necesitamos tener presente la definiciones de:
\begin{itemize}
    \item \textbf{Consecuencia Lógica}
    \item \textbf{Principio de Refutación}
    \item \textbf{Correctud de Argumentos Lógicos}
\end{itemize}
Las definiciones son las siguientes:
\newline 
\textbf{Consecuencia Lógica}
\newline
Sean $\Gamma$ un conjunto de f\'ormulas y $\varphi$ una f\'ormula. Decimos 
que $\varphi$ es consecuencia l\'ogica de $\Gamma$ si para toda interpretaci\'on 
$\mathcal{I}$ que sastisface a $\Gamma$ se tiene $\mathcal{I}(\varphi)=1$. 
Es decir si se cumple que siempre que $\mathcal{I}$ satisface a $\Gamma$ entonces 
necesariamente $\mathcal{I}$ satisface a $\varphi$.
En tal caso escribimos $\Gamma\models\varphi$.
\newline 
\textbf{Principio de Refutación}
\newline
$\Gamma\models\varphi$ sí y sólo sí $\Gamma\cup\{\lnot\varphi\}$
\newline
\textbf{Correctud de Argumentos Lógicos}
\newline 
Un argumento l\'ogico es una sucesi\'on de f\'ormulas 
$\varphi_1,\ldots,\varphi_n$ llamadas premisas y una f\'ormula $\psi$ llamada 
conclusi\'on.
\newline
Un argumento con premisas $\varphi_1,\ldots,\varphi_n$ y conclusi\'on $\psi$ es
l\'ogicamente correcto si la conclusi\'on se sigue l\'ogicamente de las 
premisas, es decir si $\{\varphi_1,\ldots,\varphi_n\}\models\psi$.
\newline
Las definiciones anteriores nos indican como demostrar de manera formal la \textbf{correctud de un
argumento}. Sea \(\Gamma = \{\varphi_1, \varphi_2, \dotsc, \varphi_n\}\), el conjunto que contiene
a las hipótesis del argumento lógico y sea \(\psi\) la conclusión entonces para demostrar que el 
argumento es correcto, existen dos maneras
\begin{itemize}
    \item Iniciamos suponiendo que \(\mathcal{I}\left(\Gamma\right) = 1\) y mediante la aplicación 
    de las definiciones de \textbf{interpretación de fórmulas} llegamos a que se cumple 
    \(\mathcal{I}\left(\psi\right) = 1\).
    \item Definimos a un conjunto \(\Gamma'\) el cual está dado por \(\Gamma \cup \{\neg \psi\}\)
    y suponiendo que \(\mathcal{I}\left(\Gamma'\right) = 1\) llegar a una contradicción, es decir,
    es satisfacible \(\Gamma'\) pero sí una variable proposicional tiene que ser verdadera y 
    falsa al mismo tiempo, lo cual sabemos que es imposible.
\end{itemize}
Específicamente el ejercicio nos pide utilizar \textbf{principio de refutación}, entonces utilizaremos
la segunda forma.
\newline 
Veamos (1)
\newline
Sea \(\Gamma = \{ p \lor q, \lnot r \to \lnot p  \}\). Tomemos la negación de \(q \to \lnot r\), es decir
\begin{align*}
    \neg \left(q \to \lnot r\right) &\equiv \neg \left(\neg q \lor \lnot r\right)  &\text{Eliminación de \(\to\)} \\
                                    &\equiv \neg \neg q \land \neg \lnot r  &\text{De Morgan} \\
                                    &\equiv q \land r &\text{Doble Negación} \\
\end{align*}
Definimos a \(\Gamma'\) como sigue:
\begin{align*}
    \Gamma \cup \{ q \land r \} &= \{ p \lor q, \lnot r \to \lnot p  \} \cup \{ q \land r \} \\
                                &= \{ p \lor q, \lnot r \to \lnot p, q \land r \} \\
\end{align*}
Supongamos que \(\mathcal{I}\left(\Gamma'\right) = 1\), es decir
\begin{align*}
    \mathcal{I}\left(p \lor q\right) &= 1 \\
    \mathcal{I}\left(\lnot r \to \lnot p\right) &= 1 \\
    \mathcal{I}\left(q \land r\right) &= 1
\end{align*}
Sabemos que \(\mathcal{I}\left(q \land r\right) = 1\) sí y sólo sí
\[
    \mathcal{I}\left(q\right) = \mathcal{I}\left(r\right) = 1
\]
Observamos que independientemente del valor que tome \(\mathcal{I}\left(p\right)\), el conjunto
es satisfacible, no llegamos a ninguna contradicción. Por lo tanto el argumento no es correcto.
\newline
Veamos (2)
\newline
Sea \(\Gamma = \{  p \to q \lor r, r \to \lnot p, q \to \lnot p \}\). Tomemos la negación de 
\(\lnot (p \land \lnot s)\), es decir
\begin{align*}
    \neg \lnot (p \land \lnot s) &\equiv p \land \lnot s  &\text{Eliminación de \(\to\)}
\end{align*}
Definimos a \(\Gamma'\) como sigue:
\begin{align*}
    \Gamma \cup \{ p \land \lnot s \} &= \{p \to q \lor r, r \to \lnot p, q \to \lnot p  \} \cup \{ p \land \lnot s \} \\
                                &= \{ p \to q \lor r, r \to \lnot p, q \to \lnot p, p \land \lnot s \} \\
\end{align*}
Supongamos que \(\mathcal{I}\left(\Gamma'\right) = 1\), es decir
\begin{align*}
    \mathcal{I}\left(p \to q \lor r\right) &= 1 \\
    \mathcal{I}\left(r \to \lnot p\right) &= 1 \\
    \mathcal{I}\left(q \to \lnot p\right) &= 1 \\
    \mathcal{I}\left(p \land \lnot s\right) &= 1
\end{align*}
Sabemos que \(\mathcal{I}\left(p \land \lnot s\right) = 1\) sí y sólo sí
\[
    \mathcal{I}\left(p\right) = \mathcal{I}\left(\lnot s\right) = 1
\]
Sabemos que \(\mathcal{I}\left(p\right) = 1\) sí y sólo sí
\[
    \mathcal{I}\left(\neg p\right) = 0
\]
Notemos que \(\mathcal{I}\left(r\right) = 0\) ya que
\begin{align*}
    \mathcal{I}\left(r \to \lnot p\right) &= 1 \\
    \mathcal{I}\left(\neg p\right) &= 0
\end{align*}
Notemos que \(\mathcal{I}\left(q\right) = 0\) ya que
\begin{align*}
    \mathcal{I}\left(q \to \lnot p\right) &= 1 \\
    \mathcal{I}\left(\neg p\right) &= 0
\end{align*}
Notemos que \(\mathcal{I}\left(p \to q \lor r\right) = 0\) ya que
\begin{align*}
    \mathcal{I}\left(p\right) &= 1 \\
    \mathcal{I}\left(q\right) &= 0 \\
    \mathcal{I}\left(r\right) &= 0
\end{align*}
Notemos que para que el conjunto sea satisfacible se tiene que cumplir que
\begin{align*}
    \mathcal{I}\left(p \to q \lor r\right) &= 1 \\
    \mathcal{I}\left(p \to q \lor r\right) &= 0
\end{align*}
Por lo tanto \(\Gamma'\) no es satisfacible. Consecuentemente
\[
    \Gamma \models \lnot (p \land \lnot s)
\]
Por lo tanto el argumento es correcto.
\subsection{Ejercicio 6}
Demuestre las siguientes propiedades de la consecuencia l\'ogica:
\begin{enumerate}
    \item Si $\Gamma,\phi\models \psi$ entonces $\Gamma,\neg\psi\models \neg \phi$.
    \item Si $\Gamma,\phi\lor\psi\models \neg\phi$ entonces
    $\Gamma,\phi\lor\psi\models\psi$.
    \item $\Gamma,\phi\land\psi\models \phi$
    \item $\Gamma,\psi\models \psi\lor\phi$
\end{enumerate}
\subsubsection{Solución}
Necesitamos tener presente la definición de:
\begin{itemize}
    \item \textbf{Consecuencia Lógica}
\end{itemize}
\textbf{Consecuencia Lógica}
\newline
Sean $\Gamma$ un conjunto de f\'ormulas y $\varphi$ una f\'ormula. Decimos 
que $\varphi$ es consecuencia l\'ogica de $\Gamma$ si para toda interpretaci\'on 
$\mathcal{I}$ que sastisface a $\Gamma$ se tiene $\mathcal{I}(\varphi)=1$. 
Es decir si se cumple que siempre que $\mathcal{I}$ satisface a $\Gamma$ entonces 
necesariamente $\mathcal{I}$ satisface a $\varphi$.
En tal caso escribimos $\Gamma\models\varphi$.
\newline 
Al igual que las siguientes equivalencias:
\begin{itemize}
    \item $\Gamma\models\varphi\to\psi$ sí y sólo sí $\Gamma\cup\{\varphi\}\models\psi$.
    \item \(\varphi \to \psi \equiv \neg \psi \to \neg \varphi \)
\end{itemize}
Veamos (1)
\begin{proof}
    Supongamos que \(\Gamma,\phi\models \psi\), sabemos que lo anterior es equivalente a
    \[
        \Gamma \models \phi \to \psi  
    \]
    Por definición de \textbf{consecuencia lógica} tenemos que
    \[
        \mathcal{I}\left(\Gamma\right) = 1 \to \mathcal{I}\left(\phi \to \psi\right) = 1
    \]
    Sabemos que \(\phi \to \psi \equiv \neg \psi \to \neg \phi\) así
    \[
        \mathcal{I}\left(\neg \psi \to \neg \psi\right) = 1
    \]
    Reescribiendo
    \[
        \mathcal{I}\left(\Gamma\right) = 1 \to \mathcal{I}\left(\neg \psi \to \neg \phi\right) = 1    
    \]
    por definición de \textbf{consecuencia lógica} tenemos que
    \[
        \Gamma \models \neg \psi \to \neg \phi
    \]
    Lo anterior es equivalente a
    \[
        \Gamma, \neg \psi \models \neg \phi
    \]
\end{proof}
Veamos (2)
\begin{proof}
    Supongamos que $\Gamma,\phi\lor\psi\models \neg\phi$. Por definición de \textbf{consecuencia 
    lógica} tenemos que
    \[
        \mathcal{I}\left(\Gamma \land \left(\phi\lor\psi\right)\right) = 1
    \]
    Sabemos que \(\mathcal{I}\left(\Gamma \land \left(\phi\lor\psi\right)\right) = 1\) sí y sólo sí
    \[
        \mathcal{I}\left(\Gamma\right) = \mathcal{I}\left(\phi\lor\psi\right) = 1  
    \]
    Sabemos que como lo anterior se cumple \textbf{(por suposición)} entonces se cumple que
    \[
        \mathcal{I}\left(\neg \phi\right) = 1
    \]
    Sabemos que \(\mathcal{I}\left(\neg \phi\right) = 1\) sí y sólo sí
    \[
        \mathcal{I}\left(\phi\right) = 0
    \]
    Sabemos que \(\mathcal{I}\left(\phi\lor\psi\right) = 1\) sí y sólo sí
    \[
        \mathcal{I}\left(\phi\right) = 1 \ \text{ó} \ \mathcal{I}\left(\psi\right) = 1
    \]
    Sabemos que \(\mathcal{I}\left(\phi\right) = 0\) entonces
    \[
        \mathcal{I}\left(\psi\right) = 1
    \]
    Ya que \(\mathcal{I}\left(\phi\lor\psi\right) = 1\).
    \newline
    Así 
    \[
        \mathcal{I}\left(\Gamma \land \left(\phi\lor\psi\right)\right) = 1 \to \mathcal{I}\left(\psi\right) = 1
    \]
    Consecuentemente, por la definición de \textbf{consecuencia lógica}
    \[
        \Gamma, \phi\lor\psi \models \psi
    \]
\end{proof}
Veamos (3)
\begin{proof}
    Supongamos que \(\mathcal{I}\left(\Gamma \land \left(\phi \land \psi\right)\right) = 1\). 
    \newline 
    Sabemos que \(\mathcal{I}\left(\Gamma \land \left(\phi \land \psi\right)\right) = 1\) sí y sólo sí
    \[
        \mathcal{I}\left(\Gamma \right) = 
        \mathcal{I}\left(\phi \land \psi\right) = 1
    \]
    Sabemos que \(\mathcal{I}\left(\phi \land \psi\right) = 1\) sí y sólo sí
    \[
        \mathcal{I}\left(\phi\right) = \mathcal{I}\left(\psi\right) = 1
    \]
    Consecuentemente tenemos que
    \[
        \mathcal{I}\left(\Gamma \land \left( \phi \land \psi \right)\right) = 1 \to 
        \mathcal{I}\left(\phi\right) = 1
    \]
    Por definición de \textbf{consecuencia lógica}
    \[
        \Gamma, \phi \land \psi \models \phi
    \]
\end{proof}
Veamos (4)
\begin{proof}
    Supongamos que \(\mathcal{I}\left(\Gamma \land \psi\right) = 1\). 
    \newline
    Sabemos que \(\mathcal{I}\left(\Gamma \land \psi\right) = 1\) sí y sólo sí
    \[
        \mathcal{I}\left(\Gamma\right) = \mathcal{I}\left(\psi\right) = 1
    \]
    Sea \(\phi \in \text{\texttt{PROP}}\) sabemos que 
    \[
        \mathcal{I}\left(\psi \lor \phi\right) = 1
    \]
    Ya que
    \[
        \mathcal{I}\left(\psi\right) = 1        
    \]
    Consecuentemente tenemos que
    \[
        \mathcal{I}\left(\Gamma \land \psi \right) = 1 \to 
        \mathcal{I}\left(\psi \lor \phi\right) = 1
    \]
    Por definición de \textbf{consecuencia lógica}
    \[
        \Gamma, \psi \models \psi \lor \phi
    \]
\end{proof}
\end{document}